\documentclass[10pt,a4paper]{article}
\usepackage[italian]{babel}
\usepackage{lmodern}
\usepackage[T1]{fontenc}
\usepackage[utf8]{inputenc}
\usepackage{pdfpages}
\usepackage{palatino}
\usepackage[left=2cm, right=2cm]{geometry}

%\usepackage[a-1b]{pdfx}
\usepackage[pdftex,
            pdfauthor={Andrea Michele Albonico},
            pdftitle={Un sistema di recommendation per la cybersecurity basato su Collaborative Filter}]{hyperref}


\begin{document}

\title{\textbf{Un sistema di recommendation per la cybersecurity basato su Collaborative Filter}}
\author{Andrea Michele Albonico (matricola 886667)}
\date{/02/2020}

\maketitle

\vspace{0.5 cm}

\begin{minipage}{\linewidth}
    \begin{tabular}{l r}
        \begin{minipage}[t]{.4\linewidth}
            \begin{flushleft}
                {
                    RELATORE\\[.15cm]
                    Prof. Valerio Bellandi
                }
            \end{flushleft}
        \end{minipage}
        &
        \begin{minipage}[t]{.53\linewidth}
            \begin{flushright}
                {
                    CORRELATORE\\[.15cm]
                    Prof. Claudio A. Ardagna\\[.1cm]
                }
            \end{flushright}
        \end{minipage}
    \end{tabular}
\end{minipage}

\vspace{2 cm}

\hfill\break
Il Cloud Computing è ormai diventato il paradigma dominante nell'ICT, tuttavia permangono problematiche legate alla mancanza di fiducia e trasparenza.
Tali criticità ancora rendono gli utenti esitanti nel migrare completamente a questo nuovo approccio.
% Infatti, spesso i fornitori di questi servizi non fornisco ai clienti le specifiche riguardanti le misure di sicurezza messe in atto.\hfill\break
Una delle strategie utilizzate per affrontare questa mancanza di fiducia è la \textit{Security Assurance}, ovvero un insieme di tecniche
per la verifica che un certo sistema ICT rispetti o meno delle proprietà di sicurezza. % mediante la raccolta continua di evidenza presso tali sistemi.
% che venga accertata la validità e l'efficienza delle proprietà di sicurezza messe in atto.\hfill\break
%
Moon Cloud è un framework di \textit{Security Assurance}%  il quale garantisce che un sistema ICT soddisfi certi requisiti
che funziona raccogliendo continuamente evidenze presso i sistemi oggetto di verifica.
% prestabiliti da appropriate politiche e procedure precedentemente definite. Una \textit{Security Compliance Evaluation} è un processo di 
% verifica a cui un target è sottoposto e il cui risultato deve soddisfare i requisiti richiesti da standard e politiche. 
% Per Evaluation si intende quel processo di verifica di uniformità di un certo target o asset, fornito dall'utente, a una o più politiche 
% attraverso una serie di Controlli che a seconda delle caratteristiche e proprietà del target, può avere successo o meno. In altre parole, 
% si può dire che un Evaluation è costituita da uno o più Controlli.\hfill\break
%
Il lavoro di tesi ha l'obiettivo di rendere le attività di \textit{Security Assurance} di Moon Cloud accessibili anche agli utenti meno esperti,
mediante lo sviluppo di un sistema di raccomandazione di verifiche di sicurezza. Il lavoro svolto si può articolare come segue.
\begin{enumerate}
    \item Studio della piattaforma Moon Cloud, in particolare dei concetti di \textit{Controlli} ed \textit{Evaluation}.
    Essi sono i due componenti base del processo di verifica di Moon Cloud, sui quali si intendono effettuare le raccomandazioni.
    \item Studio dei diversi approcci utilizzabili per la catalogazione di \textit{Controlli} ed \textit{Evaluation} in un database
    relazionale, per creare delle tassonomie. In particolare è stato necessario trovare il modo migliore per memorizzare tali tassonomie all'interno
    di un modello di dati di tipo relazionale.
    \item Studio delle diverse tipologie dei sistemi di raccomandazione e valutazione di quali fossero i più adeguati per il problema in questione.
    Si sono analizzati in particolare i sistemi \textit{Content-based filter} e i \textit{Collaborative filter} % aggiungi qualcosa
    \item Creazione di un microservizio che implementa un sistema di raccomandazione. Tale componente è in grado di offrire raccomandazioni basandosi
    su $i)$ la tipologia di target obiettivo della verifica di sicurezza, $ii)$ AGGIUNGI TU.
    Il servizio inoltre offre una serie di API per facilitare il mantenimento della coerenza tra il database principale di Moon Cloud e quello
    usato dal servizio stesso.
    % \item Analisi di diversi approcci per la creazione di tassonomie all'interno di un database relazionale unitamente a come costruire un 
    % sistema di raccomandazione. Vennero approfonditi diversi approcci, come ad esempio i \textit{Content-based filter} e i \textit{Collaborative filter}, 
    % utilizzati per filtrare i dati e determinare gli \textit{item} più adatti ad essere raccomandati all'utente.
    % \item integrazione questi due sistemi attraverso servizi di API REST che permettessero a Moon Cloud di effettuare delle richieste al sistema di 
    % raccomandazione e di ricevere come risposta una lista di Evaluation da proporre all'utente.
    % \item aggiunta di ulteriori API REST per fare in modo che il database utilizzato dalla piattaforma stessa e quello creato in questa soluzione 
    % avessero dati consistenti.
\end{enumerate}
%
Al termine dello sviluppo, la soluzione offre delle raccomandazioni di tipo basico, tuttavia è in grado di supportare gli utenti nell'utilizzo del framework 
Moon Cloud.
% sviluppi?
\end{document}
