\documentclass[10pt,a4paper]{article}
\usepackage[italian]{babel}
\usepackage{lmodern}
\usepackage[T1]{fontenc}
\usepackage[utf8]{inputenc}
\usepackage{pdfpages}
\usepackage{palatino}
\usepackage[left=2cm, right=2cm]{geometry}

%\usepackage[a-1b]{pdfx}
\usepackage[pdftex,
            pdfauthor={Andrea Michele Albonico},
            pdftitle={Un sistema di recommendation per la cybersecurity basato su Collaborative Filter}]{hyperref}


\begin{document}

\title{\textbf{Un sistema di recommendation per la cybersecurity basato su Collaborative Filter}}
\author{Andrea Michele Albonico (matricola 886667)}
\date{/02/2020}

\maketitle

\vspace{0.5 cm}

\begin{minipage}{\linewidth}
    \begin{tabular}{l r}
        \begin{minipage}[t]{.4\linewidth}
            \begin{flushleft}
                {
                    RELATORE\\[.15cm]
                    Prof. Valerio Bellandi
                }
            \end{flushleft}
        \end{minipage}
        &
        \begin{minipage}[t]{.53\linewidth}
            \begin{flushright}
                {
                    CORRELATORE\\[.15cm]
                    Prof. Claudio A. Ardagna\\[.1cm]
                }
            \end{flushright}
        \end{minipage}
    \end{tabular}
\end{minipage}

\vspace{2 cm}

\hfill\break
Il Cloud Computing è ormai diventato il paradigma dominante nell'ICT, tuttavia permangono problematiche legate alla mancanza di fiducia e trasparenza.
Tali criticità ancora rendono gli utenti esitanti nel migrare completamente a questo nuovo approccio. Infatti, spesso i fornitori di questi servizi 
non fornisco ai clienti le specifiche riguardanti le misure di sicurezza messe in atto.\hfill\break
Una delle strategie utilizzate per affrontare questa mancanza di fiducia è la \textit{Security Assurance}, ovvero un insieme di tecniche
per la verifica che un certo sistema ICT rispetti o meno delle proprietà di sicurezza; grazie alla raccolta e allo studio di \textit{evidence} è possibile 
che venga accertata la validità e l'efficienza delle proprietà di sicurezza messe in atto.\hfill\break
%
Moon Cloud è un framework di \textit{Security Assurance} il quale garantisce che un sistema ICT soddisfi certi requisiti 
prestabiliti da appropriate politiche e procedure precedentemente definite. Una \textit{Security Compliance Evaluation} è un processo di 
verifica a cui un target è sottoposto e il cui risultato deve soddisfare i requisiti richiesti da standard e politiche. 
Per Evaluation si intende quel processo di verifica di uniformità di un certo target o asset, fornito dall'utente, a una o più politiche 
attraverso una serie di Controlli che a seconda delle caratteristiche e proprietà del target, può avere successo o meno. In altre parole, 
si può dire che un Evaluation è costituita da uno o più Controlli.\hfill\break
%
Tali verifiche di sicurezza dovrebbero essere svolte in maniera automatizzata, e fornire un supporto completo agli utenti, anche ai meno esperti.
In questo contesto si inserisce il lavoro di tesi, che si può articolare nei seguenti come segue.
\begin{itemize}
    \item studio del funzionamento della piattaforma Moon Cloud, in particolare vennero raccolte le Evaluation e i Controlli implementati;
    \item analisi di diversi approcci per la creazione di tassonomie all'interno di un database relazionale unitamente a come costruire un 
    sistema di raccomandazione. Vennero approfonditi diversi approcci, come ad esempio i \textit{Content-based filter} e i \textit{Collaborative filter}, 
    utilizzati per filtrare i dati e determinare gli \textit{item} più adatti ad essere raccomandati all'utente.
    \item integrazione questi due sistemi attraverso servizi di API REST che permettessero a Moon Cloud di effettuare delle richieste al sistema di 
    raccomandazione e di ricevere come risposta una lista di Evaluation da proporre all'utente.
    \item aggiunta di ulteriori API REST per fare in modo che il database utilizzato dalla piattaforma stessa e quello creato in questa soluzione 
    avessero dati consistenti.
\end{itemize}
%
Al termine dello sviluppo, la soluzione offre delle raccomandazioni di tipo basico, tuttavia è in grado di supportare gli utenti nell'utilizzo del framework 
Moon Cloud.
\end{document}
