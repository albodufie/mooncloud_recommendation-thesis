\documentclass[10pt,a4paper]{article}
\usepackage[italian]{babel}
\usepackage{lmodern}
\usepackage[T1]{fontenc}
\usepackage[utf8]{inputenc}
\usepackage{pdfpages}
\usepackage{palatino}
\usepackage[left=2cm, right=2cm]{geometry}

%\usepackage[a-1b]{pdfx}
\usepackage[pdftex,
            pdfauthor={Andrea Michele Albonico},
            pdftitle={Sistema di raccomandazione basato su Collaborative Filter per piattaforma Moon Cloud
            facente parte dell'ambito della Security Assurance}]{hyperref}


\begin{document}

\title{\textbf{Sistema di raccomandazione basato su Collaborative Filter per piattaforma Moon Cloud
facente parte dell'ambito della Security Assurance}}
\author{Andrea Michele Albonico (matricola 886667)}
\date{/02/2020}

\maketitle

\vspace{0.5 cm}

\begin{minipage}{\linewidth}
    \begin{tabular}{l r}
        \begin{minipage}[t]{.4\linewidth}
            \begin{flushleft}
                {
                    RELATORE\\[.15cm]
                    Prof. Claudio A. Ardagna
                }
            \end{flushleft}
        \end{minipage}
        &
        \begin{minipage}[t]{.53\linewidth}
            \begin{flushright}
                {
                    CORRELATORE\\[.15cm]
                    Prof. Valerio Bellandi\\[.1cm]
                }
            \end{flushright}
        \end{minipage}
    \end{tabular}
\end{minipage}

\vspace{2 cm}

\hfill\break
Il Cloud Computing ha portato un rivoluzionario paradigma nella creazione di un nuovo business, virtuale e accessibile, 
in qualunque momento e luogo; esso sfrutta le tecnologie messe a disposizione dai sistemi ICT come le operazioni di virtualized computing, 
internet e distributed computing, provvedendo un sistema integrato molto potente. Si può definire il Cloud Computing come l'abilità di accedere a 
risorse (come database o applicazioni) in poco tempo e in tutto il mondo attraverso una rete.\hfill\break
Gli immensi benefici del Cloud in termini di flessibilità, consumo delle risorse e gestione semplificata, lo rendono la prima scelta per 
utenti e industrie per il deploy dei loro sistemi IT. Tuttavia il Cloud Computing solleva diverse problematiche legate alla mancanza di 
fiducia e trasparenza dove i clienti necessitano di avere delle garanzie sui servizi Cloud ai quali si affidano; spesso i fornitori di 
questi servizi non fornisco ai clienti le specifiche riguardanti le misure di sicurezza messe in atto.\hfill\break
Per rendere il Cloud fidato e trasparente, per questo sono state introdotte tecniche di \textit{Security Assurance}, delle garanzie che 
permettono di ottenere la fiducia necessaria nelle infrastrutture e/o nelle applicazioni di dimostrare il rispetto di certe proprietà 
di sicurezza, e che operino normalmente anche se subiscono attacchi; grazie alla raccolta e allo studio di evidence è possibile che 
venga accertata la validità e l'efficienza delle proprietà di sicurezza messe in atto.\hfill\break
Moon Cloud è un framework di \textit{Security Assurance} il quale garantisce che un sistema ICT soddisfi certi requisiti 
prestabiliti da appropriate politiche e procedure precedentemente definite. Una \textit{Security Compliance Evaluation} è un processo di 
verifica a cui un target è sottoposto e il cui risultato deve soddisfare i requisiti richiesti da standard e politiche. 
Per Evaluation si intende quel processo di verifica di uniformità di un certo target o asset, fornito dall'utente, a una o più politiche 
attraverso una serie di Controlli che a seconda delle caratteristiche e proprietà del target, può avere successo o meno. In altre parole, 
si può dire che un Evaluation è costituita da uno o più Controlli. 
I sistemi di raccomandazione (\textit{Recommendation System}) sono nati con lo scopo d'identificare quegli oggetti (detti generalmente \textit{item}) 
all'interno di un vasto mondo d'informazioni che possono essere di nostro interesse e tanto maggiore è il grado di conoscenza dell'individuo 
e tanto più vengono ritenuti affidabili.\hfill\break
Per poter rendere ancora più intuitivo e semplice da utilizzare un sistema di questa importanza, si è pensato d'introdurre un sistema 
che possa raccomandare agli utenti, in base agli asset che vogliono proteggere e monitorare, una serie di Evaluation o politiche da 
applicare in quei casi; questo permette anche a utenti meno esperti di poter configurare in modo rapido ed efficiente meccanismi di 
protezione da minacce. Un sistema di raccomandazione permette di selezionare all’interno di un ampio catalogo, un numero limitato di prodotti 
personalizzati sulla base delle preferenze dell’utente attivo. La ricerca in questo ambito si è sempre concentrata sulla qualità delle 
raccomandazioni di questi sistemi, tralasciando un aspetto fondamentale: la fiducia che un utente deve avere verso questi ultimi. 
E ciò è ottenibile se si è il più possibile trasparenti nei processi che portano alla nascita dei suggerimenti, partendo da questo 
si può ottenere l'ambita fiducia da parte degli utenti.
\end{document}