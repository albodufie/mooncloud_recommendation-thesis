\documentclass[10pt,a4paper]{article}
\usepackage[italian]{babel}
\usepackage{lmodern}
\usepackage[T1]{fontenc}
\usepackage[utf8]{inputenc}
\usepackage{pdfpages}
\usepackage{palatino}
\usepackage[left=2cm, right=2cm]{geometry}

%\usepackage[a-1b]{pdfx}
\usepackage[pdftex,
            pdfauthor={Andrea Michele Albonico},
            pdftitle={Un sistema di recommendation per la cybersecurity basato su Collaborative Filter}]{hyperref}


\begin{document}

\title{\textbf{Un sistema di recommendation per la cybersecurity basato su Collaborative Filter}}
\author{Andrea Michele Albonico (matricola 886667)}
\date{/02/2020}

\maketitle

\vspace{0.5 cm}

\begin{minipage}{\linewidth}
    \begin{tabular}{l r}
        \begin{minipage}[t]{.4\linewidth}
            \begin{flushleft}
                {
                    RELATORE\\[.15cm]
                    Prof. Valerio Bellandi
                }
            \end{flushleft}
        \end{minipage}
        &
        \begin{minipage}[t]{.53\linewidth}
            \begin{flushright}
                {
                    CORRELATORE\\[.15cm]
                    Prof. Claudio A. Ardagna\\[.1cm]
                }
            \end{flushright}
        \end{minipage}
    \end{tabular}
\end{minipage}

\vspace{2 cm}

\hfill\break
Il Cloud Computing solleva diverse problematiche legate alla mancanza di fiducia e trasparenza visto che i clienti necessitano di avere delle 
garanzie sui servizi Cloud ai quali si affidano; spesso i fornitori di questi servizi non fornisco ai clienti le specifiche riguardanti le misure 
di sicurezza messe in atto.\hfill\break
Per rendere il Cloud fidato e trasparente, per questo sono state introdotte tecniche di \textit{Security Assurance}, delle garanzie che 
permettono di ottenere la fiducia necessaria nelle infrastrutture e/o nelle applicazioni di dimostrare il rispetto di certe proprietà 
di sicurezza; grazie alla raccolta e allo studio di evidence è possibile che venga accertata la validità e l'efficienza delle proprietà di 
sicurezza messe in atto.\hfill\break
%% LA PARTE QUI SOPRA SI POTREBBE TOGLIERE (O ACCORCIARE) E PARTIRE SUBITO CON LA DESCRIZIONE DI MOON CLOUD (FORSE IN QUESTO MODO PUò ANDARE) %%
Moon Cloud è un framework di \textit{Security Assurance} il quale garantisce che un sistema ICT soddisfi certi requisiti 
prestabiliti da appropriate politiche e procedure precedentemente definite. Una \textit{Security Compliance Evaluation} è un processo di 
verifica a cui un target è sottoposto e il cui risultato deve soddisfare i requisiti richiesti da standard e politiche. 
Per Evaluation si intende quel processo di verifica di uniformità di un certo target o asset, fornito dall'utente, a una o più politiche 
attraverso una serie di Controlli che a seconda delle caratteristiche e proprietà del target, può avere successo o meno. In altre parole, 
si può dire che un Evaluation è costituita da uno o più Controlli.\hfill\break
I sistemi di raccomandazione (\textit{Recommendation System}) sono nati con lo scopo d'identificare quegli oggetti (detti generalmente \textit{item}) 
all'interno di un vasto mondo d'informazioni che possono essere di nostro interesse.\hfill\break
Per poter rendere ancora più intuitivo e semplice da utilizzare un sistema di questa importanza, si è pensato d'introdurre un sistema che 
possa raccomandare agli utenti, in base agli asset che vogliono proteggere e monitorare, una serie di Evaluation o politiche da applicare in quei casi; 
questo permette anche a utenti meno esperti di poter configurare in modo rapido ed efficiente meccanismi di protezione da minacce.\hfill\break
Inizialmente lo sviluppo di questa soluzione ha affrontato una fase di studio del funzionamento della piattaforma Moon Cloud, 
in particolare vennero raccolte le Evaluation e i Controlli implementati; si analizzarono i diversi approcci 
per la creazione di tassonomie all'interno di un database relazionale unitamente a come costruire un sistema di raccomandazione. Vennero approfonditi diversi 
approcci, come ad esempio i \textit{Content-based filter} e i \textit{Collaborative filter}, utilizzati per filtrare i dati e determinare gli \textit{item} 
più adatti ad essere raccomandati all'utente. Successivamente si è passati a integrare quanto fatto con la piattaforma già esistente. Per integrare 
questi due sistemi vennero predisposti dei servizi di API REST che permettessero a 
Moon Cloud di effettuare delle richieste al sistema di raccomandazione e di ricevere come risposta una lista di Evaluation da proporre all'utente.
Inoltre per fare in modo che il database utilizzato dalla piattaforma stessa e quello creato in questa soluzione avvessero dati consistenti vennero predisposte 
ulteriori API REST.\hfill\break
In conclusione, la soluzione sviluppata offre delle raccomandazioni di tipo basico, tuttavia è in grado di supportare gli utenti nell'utilizzo del framework 
Moon Cloud.
\end{document}
