% SLIDE 1: pagina di presentazione
\frame{\titlepage}

% SLIDE 2: motivazioni
\begin{frame}
    \frametitle{Scenario e Motivazioni}
    I sistemi IT attuali sono \alert{complessi} ed \alert{eterogenei}, attività di \alert{\textit{Security Assurance}} diventano fondamentali 
    \begin{itemize}
        \item numerosi strumenti di sicurezza esistenti
        \item spesso \alert{complessi} da utilizzare
        \item difficilmente accessibili a utenti poco esperti
    \end{itemize}
    % Il mondo del Cloud in cooperazione con i sistemi IT ha portato molti benefici, tuttavia vengono sollevate diverse problematiche
    % \begin{itemize}
    %     \item questi sistemi sono spesso \alert{complessi e impegnativi da gestire} se non si hanno conoscenze approfondite in materia, 
    %     soprattutto per utenti poco esperti
    % \end{itemize}
\end{frame}

% SLIDE 3: obiettivo della tesi
\begin{frame}
    \frametitle{Obiettivi}
    Sviluppare un \alert{sistema di raccomandazione} di \alert{verifiche di sicurezza} che consenta di:
    \begin{itemize}
        \item suggerire agli utenti quali sono le verifiche \alert{più adatte} per i loro sistemi
        \vspace{-0.4 cm}
        \begin{itemize}
            \item sulla base di diversi parametri
        \end{itemize}
        \item fornire un \alert{supporto} alle \alert{attività di \textit{Security Assurance}}
    \end{itemize}
    % Introduzione di un \alert{sistema di raccomandazione} che possa consigliare all'utente delle possibili \textit{Evaluation} rispetto 
    % al \textit{target} che vuole proteggere e monitorare
    % \begin{itemize}
    %     \item l'utente meno esperto può usufruire dei servizi offerti da Moon Cloud in modo \alert{efficace} ed \alert{intuitivo} 
    %     \item viene supportanto in ogni fase dall'inserimento delle informazioni del target da proteggere fino alla scelta delle 
    %     attività di monitoraggio da eseguire
    % \end{itemize}
\end{frame}

% SLIDE 4: Moon Cloud
\begin{frame}
    \frametitle{Moon Cloud}
    \begin{columns}
        \begin{column}{0.55\textwidth}
            Framework di \alert{\textit{Security Assurance}}
            \begin{itemize}
                \item erogato \alert{as-a-Service}
                \begin{itemize}
                    \item l'utente configura le verifiche (\alert{\textit{Evaluation}}) di sicurezza
                    \item Moon Cloud le esegue e mostra risultati
                \end{itemize}
                \item controllo \alert{centralizzato} della sicurezza di un sistema
                %\item l'utente conf (\alert{\textit{Evaluation}})
            \end{itemize}
        \end{column}
        \begin{column}{0.4\textwidth}
            \begin{center}
                \includegraphics[scale=0.12]{images/mc}
            \end{center}
        \end{column}
    \end{columns}
    % Garantisce il controllo della sicurezza informatica in modo rapido ed efficiente, attraverso attività di test e monitoraggio 
    % periodiche e programmate
\end{frame}

% SLIDE 5: Sistema di Raccomandazione
\begin{frame}
    \frametitle{Sistemi di raccomandazione}
    Un \textit{recommendation system} filtra i dati usando diversi algoritmi e raccomanda gli item più rilevanti agli utenti attraverso 
    un procedimento a 3 fasi
    \begin{enumerate}
        \item \alert{raccolta di dati}: ottenere informazioni rilevanti e consistenti su cui applicare algoritmi di raccomandazione
        \item \alert{memorizzazione di dati}: la quantità di dati definisce quanto efficace un modello di raccomandazione può diventare
        \item \alert{filtraggio dei dati}: estrarre le informazioni più rilevanti
    \end{enumerate}
\end{frame}

% SLIDE 6: Recommendation Algorithms
\begin{frame}
    \frametitle{Recommendation Algorithms}
    \begin{columns}
        \begin{column}{0.5\textwidth}
            \begin{figure}
                \centering
                \includegraphics[scale=0.5]{images/UB_CF_ex}
           \end{figure}
           \alert{User Based Collaborative Filter}: algoritmo che fornisce dei suggerimenti sulla base di uno o più vicini (\textit{neighbours}) 
        \end{column}
        \begin{column}{0.5\textwidth}
            \begin{figure}
                \centering
                \includegraphics[scale=0.55]{images/IB_CF_ex2}
            \end{figure}
            \alert{Item based Collaborative Filter}: algoritmo che confronta gli item dell'utente a cui si vuole raccomandare e i possibili item simili 
        \end{column}
    \end{columns}
\end{frame}

% SLIDE 7: Struttura del database
\begin{frame}
    \frametitle{Tassonomia}
    \begin{figure}
        \centering
        \includegraphics[scale=0.35]{images/MC_Rec_Tree2}
    \end{figure}
\end{frame}

% SLIDE 9: Soluzione
\begin{frame}
    \frametitle{Soluzione}
    Servizio di \alert{API REST} che implementa un \alert{sistema di raccomandazione} \alert{integrato} nel \alert{backend} di Moon Cloud
    % accessibile attraverso apposite URL che permette di effettuare richieste al sistema di  raccomandazione e di mantenere consistente la base di dati
    \begin{itemize}
        \item offre diversi tipi di raccomandazioni per \alert{suggerire \textit{Evaluation}}
        \begin{itemize}
            \item \alert{user-based} basato su \textit{Evaluation} passate eseguite dall'utente%, il quale si basa sulle Evaluation usate dall'utente a cui si vuole raccomandare 
            \item \alert{item-based} basato su \textit{Evaluation} o \textit{Target}%, il quale si può basare o su Evaluation o su Target forniti dall'utente 
            \item \alert{ibrido} condensa i pregi dei precedenti algoritmi%, cerca di mettere insieme i pregi dei precedenti algoritmi per determinare le Evaluation simili
        \end{itemize}
        \item garantisce la \alert{coerenza} tra il database principale di Moon Cloud e quello usato dal servizio
    \end{itemize}
\end{frame}

% SLIDE 9: Soluzione (2)
\begin{frame}
    \frametitle{Soluzione (2)}
    \begin{figure}
        \centering
        \includegraphics[scale=0.34]{images/UML_MoonCloud_HowToDoPres}
    \end{figure}
\end{frame}

% SLIDE 10: Conclusioni
\begin{frame}
    \frametitle{Conclusioni}
    La soluzione introduce un sistema di raccomandazione in un contesto in cui spesso non è presente %perché popolato da utenti esperti
    \begin{itemize}
        \item la configurazione delle \alert{attività di test} vengono \alert{semplificate} 
        \item la \textit{Security Assurance} diventa maggiormente accessibile%un maggior numero di utenti può fare uso di questi sistemi in totale sicurezza e affidabilità \textbf{OK MA PIÀ STRINGATO: ES RENDE ASSURANCE PIÙ ACCESSIBILE}
    \end{itemize}
\end{frame}
