% SLIDE 1: pagina di presentazione
\frame{\titlepage}

% SLIDE 2: motivazioni
\begin{frame}
    \frametitle{Scenario e Motivazioni}
    Il mondo del Cloud in cooperazione con i sistemi IT ha portato molti benefici, tuttavia vengono sollevate diverse problematiche
    \begin{itemize}
        \item \alert{poca fiducia} nel mondo del cloud da parte degli utenti
        \item col tempo questi sistemi sono diventati molto \alert{complessi e impegnativi da gestire}
        \item inoltre sono specifici e \alert{difficili da utilizzare} se non si ha esperienza in materia
    \end{itemize}
\end{frame}

% SLIDE 3: obiettivi
\begin{frame}
    \frametitle{Obiettivo della tesi}
    Introduzione di un \alert{sistema di raccomandazione} che possa consigliare all'utente delle possibili \textit{Evaluation} rispetto 
    all'\textit{asset} che vuole proteggere e monitorare
    \begin{itemize}
        \item l'utente meno esperto può usufruire dei servizi offerti da Moon Cloud in modo \alert{semplice} e \alert{intuitivo}
        \item si è cercato di colmare il problema della mancata fiducia in questi sistemi
    \end{itemize}
\end{frame}

% SLIDE 4: Moon Cloud
\begin{frame}
    \frametitle{Moon Cloud}
    \begin{columns}
        \begin{column}{0.55\textwidth}
            Moon Cloud è una piattaforma erogata come servizio, la quale supporta:
            \begin{itemize}
                \item un sistema di \textit{Security Governance}
                \item un framework di \alert{\textit{Security Assurance}}, basato su Controlli ed Evaluation
            \end{itemize}
        \end{column}
        \begin{column}{0.4\textwidth}
            \begin{center}
                \includegraphics[scale=0.12]{images/mc}
            \end{center}
        \end{column}
    \end{columns}
    Garantisce il controllo della sicurezza informatica in modo rapido ed efficiente, attraverso attività di test e monitoraggio 
    periodiche e programmate
\end{frame}

% SLIDE 5: Sistema di Raccomandazione
\begin{frame}
    \frametitle{Sistema di raccomandazione}
    Un \textit{recommendation system} può filtrare i dati usando differenti algoritmi e raccomandare gli item più rilevanti agli utenti attraverso 
    un procedimento a 3 fasi
    \begin{enumerate}
        \item \alert{Raccolta di dati}: ottenere informazioni rilevanti e consistenti su cui applicare algoritmi di raccomandazione
        \item \alert{Memorizzazione di dati}: la quantità di dati definisce quanto efficace un modello di raccomandazione può di diventare
        \item \alert{Filtraggio dei dati}: estrarre le informazioni più rilevanti
    \end{enumerate}
\end{frame}

% SLIDE 6: Collaborative Filter
\begin{frame}
    \frametitle{Collaborative filtering}
    Questo sistema predice la preferenza che un utente accorderebbe a un item basandosi sulle preferenze date da altri utenti
    \begin{itemize}
        \item \alert{semplici da implementare}, \alert{intuitivi} e non necessitano di operazioni di training sui dati e 
        regolazione di molti parametri
        \item permettono all'utente di comprendere le ragioni che si celano dietro ad ogni raccomandazione
    \end{itemize}
\end{frame}

% SLIDE 8: Recommendation Algorithms
\begin{frame}
    \frametitle{Recommendation Algorithms}
    \begin{columns}
        \begin{column}{0.5\textwidth}
            \begin{figure}
                \centering
                \includegraphics[scale=0.5]{images/UB_CF_ex}
           \end{figure}
           %\alert{User Based Collaborative Filter}: algoritmo che fornisce dei suggerimenti sulla base di uno o più vicini (\textit{neighbours})
        \end{column}
        \begin{column}{0.5\textwidth}
            \begin{figure}
                \centering
                \includegraphics[scale=0.5]{images/IB_CF_ex}
            \end{figure}
            %\alert{Item based Collaborative Filter}: algoritmo che confronta gli item dell'utente a cui si vuole raccomandare e i possibili item simili
        \end{column}
    \end{columns}
\end{frame}

% SLIDE 9: Soluzione
\begin{frame}
    \frametitle{Soluzione}
    Servizio di \alert{API REST} accessibile attraverso apposite URL che permette di effettuare richieste al sistema di 
    raccomandazione e di aggiornare la base di dati
    \begin{enumerate}
        \item in base alle informazioni inserite dall'utente sull'\alert{asset} da proteggere e alle Evaluation usate vengono generate delle 
        liste 
        \item queste liste contengono le Evaluation da raccomandare a quell'utente
    \end{enumerate}
\end{frame}

% SLIDE 9: Soluzione (2)
\begin{frame}
    \frametitle{Soluzione (2)}
    \begin{figure}
        %\centering
        \includegraphics[scale=0.28]{images/UML_MoonCloud_HowToDoPres}
    \end{figure}
\end{frame}

% SLIDE 10: Conclusioni
\begin{frame}
    \frametitle{Conclusioni}
    La soluzione introduce un sistema di raccomandazione in un mondo in cui spesso non è presente perché 
    popolato da utenti esperti
    \begin{itemize}
        \item la configurazione delle attività di test vengono semplificate
        \item un maggior numero di utenti può fare uso di questi sistemi in totale sicurezza e affidabilità
    \end{itemize}
\end{frame}

% SLIDE 12:
%\begin{frame}
%    \frametitle{}

%\end{frame}
