\documentclass[12pt,fleqn,twoside,a4paper]{book}

\usepackage[T1]{fontenc} % codifica dei font
\usepackage[utf8]{inputenc} % lettere accentate da tastiera
\usepackage[italian]{babel} % lingua del documento
\usepackage{float}
\usepackage{graphicx}
\graphicspath{{images/}}
\usepackage{rotating} % rotate an image
%\usepackage{hyperref} % collegamenti ipertestuali e carica anche il paccheto 'url' (caricato per ultimo)
\usepackage{listings} % inserire codice sorgente a modo
\usepackage[table]{xcolor}

% colori per le parti di codice in python
\definecolor{codeorange}{rgb}{1,0.5,0}
\definecolor{codegreen}{rgb}{0,0.7,0}
\definecolor{codegray}{rgb}{0.5,0.5,0.5}
\definecolor{codedarkgreen}{rgb}{0,0.4,0}
\definecolor{backcolour}{rgb}{0.95,0.95,0.92}

% colori per le tabelle in cui identifico il root node, i nodi categoria e i nodi evaluation/control
\definecolor{rootnodecell}{rgb}{0.68,0.25,0.19}
\colorlet{rootnodecell}{rootnodecell!50}
\definecolor{categorycell}{rgb}{0.07,0.66,0.71}
\colorlet{categorycell}{categorycell!50}
\definecolor{evaluationcell}{rgb}{0.94,0.57,0.12}
\colorlet{evaluationcell}{evaluationcell!50}

\lstdefinestyle{python_code_style}{
    backgroundcolor=\color{backcolour},
    commentstyle=\color{codegreen},
    keywordstyle=\color{codeorange},
    numberstyle=\tiny\color{codegray},
    stringstyle=\color{codedarkgreen},
    basicstyle=\ttfamily\tiny,
    breakatwhitespace=false,
    breaklines=true,
    captionpos=b,
    keepspaces=true,
    numbers=left,
    numbersep=5pt,
    showspaces=false,
    showstringspaces=false,
    showtabs=false,
    tabsize=2
}


\usepackage{csquotes}
\usepackage[backend=biber, style=numeric, sorting=nty]{biblatex}
\addbibresource{bibliography.bib}

\usepackage{imakeidx}
\makeindex

\title{Sistema di raccomandazione basato su Collaborative Filter per piattaforma Moon Cloud 
facente parte dell'ambito della Security Assurance}
\author{Andrea Michele Albonico}


\begin{document}

\frontmatter

\begin{titlepage}
    \begin{figure}
        \centering
        \includegraphics[height=5.0 cm]{logo.jpg}
        \vspace{0.5 cm}
    \end{figure}
    \begin{center}
        {\Large Corso di Laurea in Informatica Musicale}
    \end{center}
    \begin{center}
        \vspace{2 cm}
        {\Large \textsc{Sistema di raccomandazione basato su Collaborative Filter per piattaforma Moon Cloud 
        facente parte dell'ambito della Security Assurance} }
    \end{center}
    \par
    \vspace{2 cm}
    \begin{flushleft}
        Relatore:\\ Prof. Claudio Agostino Ardagna\\
        \noindent Correlatore:\\ Prof. Valerio Bellandi
    \end{flushleft}
    \vspace{1 cm}
    \begin{flushright}
        Tesi di Laurea di:\\ Andrea Michele Albonico\\ Matricola: 886667
    \end{flushright}
    \vfill
    \begin{center}
        {\Large Anno Accademico 2018/2019}
    \end{center}
\end{titlepage}

\break

% RINGRAZIAMENTI 
\chapter*{Ringraziamenti}



\begin{flushright}
    \textit{Andrea Michele Albonico}
\end{flushright}

% PREFAZIONE: motivazione con breve descrizione del progetto
\chapter{Prefazione}
I sistemi di raccomandazione (\textit{Recommendation System}) hanno avuto un forte sviluppo negli ultimi decenni e 
nascono proprio con lo scopo di identificare quegli oggetti (detti generalmente \textit{item}) all'interno di vasto 
mondo di informazioni che possono essere di nostro interesse e tanto maggiore è il grado di conoscenza dell'individuo 
e tanto più vengono ritenuti affidabili.
\newline
Il motivo di questo successo risiede nella riuscita integrazione di tali sistemi in applicazioni commerciali, 
soprattutto nel mondo dell’E-commerce e nel fatto che sono in grado di aiutare un utente a prendere una decisione 
che sia la scelta di un film per l'uscita con gli amici il sabato sera, di una playlist da ascoltare durante 
un viaggio in auto o in un momento di lettura, e via discorrendo. 
\newline
MoonCloud è una piattaforma erogata come servizio che fornisce un meccanismo di \textit{Security Governance} centralizzato. 
Garantisce il controllo della sicurezza informatica in modo semplice e intuitivo, attraverso attività di test e monitoraggio 
periodiche e programmate (\textit{Security Assurance}). L'obbiettivo di questa tesi è stato quello di aggiungere, al già 
presente sistema per la scelta dei controlli all'interno delle attività di test, un sistema di raccomandazioni che possa 
consigliare all'utente delle possibili \textit{evaluation} rispetto ai dati relativi al target indicato; in questo modo 
anche l'utente meno esperto può usufruire dei servizi offerti da MoonCloud in modo semplice e intuitivo.  


La tesi \`e organizzata come segue:
\begin{description}
    \item[Capitolo 1 -- Introduzione a MoonCloud] descrizione e funzionamento della piattaforma MoonCloud. 
    \item[Capitolo 2 -- Descrizione delle attività preliminari] studi e analisi di soluzioni esistenti, studi delle tecnologie utilizzate
    nel seguito del lavoro.
    \item[Capitolo 3 -- ] Descrizione delle attività svolte per conseguire gli obiettivi: Descrivere le attività svolte, 
    riportando attivita, tempi, strumenti utilizzati, risultati conseguiti, problemi affrontati e modalita di risoluzione. 
    Potranno essere qui descritte le attività anche dal punto di vista strettamente tecnico, approfondendo le scelte effettuate, 
    le motivazioni, le alternative prese in considerazione, l’uso o il possibile uso dei risultati del lavoro.
    \item[Capitolo 4 -- ] Presentazione dei risultati e conclusioni] La presentazione dei risultati dovrebbe consistere in una 
    descrizione tecnica dei risultati raggiunti,unitamente ad un commento critico e ad un’analisi della rispondenza agli 
    obiettivi iniziali (si consiglia pertanto di motivare 
    la rilevanza dei risultati e l’eventuale scostamento dagli obiettivi iniziali). La sezione relativa ai risultati dovrebbe 
    infine contenere una sintesi critica e un giudizio sull’esperienza effettuata, che renda conto di aspetti positivi e negativi
    per il tirocinante e per l’ente ospitante, del valore formativo, professionale e umano, così via.
\end{description}






% INDICE DELLA TESI
\tableofcontents
\listoffigures
%\listoftables


%       CONTENUTO DELLA TESI        %
\mainmatter

% 1- Stato dell'arte e introduzione con descrizione più approfondita
\chapter{Introduzione}
In questo capitolo verrà descritto in modo più approfondito 
il funzionamento della piattaforma Moon Cloud e unitamente al 
motivo dell'implementazione della soluzione proposta.

\section{Perchè Moon Cloud?}
La diffusione di sistemi ICT (\textit{Information and Communications Technology})
nella maggiorparte degli ambienti lavorativi e privati in termini di servizi offerti,
automazione di processi e incremento delle performance. L'uso di questa tecnologia 
ha assunto importanza a partire dagli anni novanta come effetto del boom di Internet.
Oggi le professionalità legate all'ICT crescono in numero e si evolvono per 
specificità, per operare in ambienti fortemente eterogenei ma sempre più interconnessi 
fra di loro come cloud computing, social newtwork, marketing digitale, IoT, realtà virtuale,
ecc.

Il prezzo che paghiamo per i benefici di queste tecnologie è dato dall'incremento di 
violazione di sicurezza, che oggigiorno preoccupa tutte le aziende, e di conseguenza anche i 
loro clienti, con l'incremento del rischio di fallimento per i servizi più importanit, 
violazioni della privacy e furto di dati.
\newline
Il mercato sta lentamente notando che non è l'inadequamento tecnologico dei sistemi di sicurezza
che incrementa il rischio di furti di dati o violazioni di sicurezza; piuttosto, la mal configurazione
e errata integrazione di questi sistemi nei processi di business sono la base per 
i furti e le violazioni. \cite{cloud-Platform-for-ICT-Security-Governance}

Per questo motivo anche se vengono usati i sistemi di sicurezza e di controllo migliori non è possibile 
garantire la sicurezza; ma è necessario implementare un processo continuo di diagnostica che verifica che 
controlli sono configurati in modo corretto e il loro comportamento è quello aspettato.

Security assesment diventa allora un aspetto importante specialmente negli ambienti cloud e IoT. Questo 
assesment deve essere fatto in modo continuo e olistico, per correlare le prove raccolte da sempre maggiori 
meccanismi di protezione
\newline
Moon Cloud è una soluzione PaaS (Platform as a Service) che fornice una piattaforma B2B innovativa per verifiche, 
diagnostiche e monitoraggio dell'adeguatezza dei sistemi ICT rispetto alle politiche di sicurezza, in modo continuo 
e su larga scala.
Moon Cloud supporta una semplice ed efficente \textit{ICT security governance}, dove le politiche di sicurezza possono
essere definite dalle compagnie stesse (a partire da un semplice controllo sulle vulnerabilità a linee guida di
sicurezza interna), da entità esterne, imposte da standard oppure da regolamentazioni nazionali/internazionali.
\newline
La sicurezza di un sistema o di un insieme di asset dipende solo parzialmente dalla forza dei
singoli meccanismi di protezione isolati l'uno dall'altro; infatti, dipende anche dall'abilità di questi meccanismi 
di lavorare continuamente in sinergia per provvedere a una protezione olistica.
In più, quando i sistemi cloud e i servizi IoT sono coinvolti, le dinamiche di questi servizi e la loro rapida 
evoluzione rende il controllo dei processi all'interno dell'azienda e le politiche di sicurezza 
più complesse e prone ad errori.
\newline
I requisiti ad alto livello fondamentali per poter garantire le security assurance sono:
\begin{description}
	\item[sistema olistico] è richiesta una visione globale e pulita dello stautus dei sistemi di sicurezza; 
	inoltre è cruciale distribuire lo sforzo degli specialisti in sicurezza per migliorare il processo e le 
	politiche messe in atto. Si parte da della valutazioni fatte manualmente a quella semi-automatiche che
	ispezionano i meccanismi di sicurezza. 
	\item[monitoraggio continuo ed efficiente] è necessario un controllo continuo che valuti l'efficenza dei 
	sistemi di sicurezza per ridurre l'impatto dell'errore umano, soprattutto dal punto di vista organizzativo.
	La mancata configurazione dovuta al cambiamento dell'ambiente, la coesistenza di componenenti in conflitto: 
	sono scenari che richiedono un monitoraggio e un aggiornamento continuo.
	\item[singolo punto managment] avere un solo punto in cui gestire tutti gli aspetti relativi alla sicurezza,
	permette di avere sotto controllo le politiche di sicurezza, Inoltre disporre di un inventario degli asset 
	da proteggere, così da poter conoscere quali protezioni applicare.
	\item[reazioni rapide a incidenti di sicurezza] spesso la reazione ad incidenti di sicurezza è ritardata 
	da due fattori: il tempo richiesto per rilevare l'incidente e il tempo per analizzare il motivo dell'accaduto.
\end{description}


Moon Cloud è basato su una tecnica di security assurance garantendo che tutti le attività aziendali si compiano
seguendo i requisiti prestabiliti da appropriate politiche e procedure.

Una security compliance evaluation è il processo di verifica che un target è sottoposto il cui risultato deve 
soddisfare i richiesti standard e politiche. Da questi processi di verifica, che devono a loro volta essere 
affidabili, si ottendo delle evidence (prove); queste ultime possono essere raccolte monitorando l'attività
del target oppure, come già menzionato, sottoponendo il target a scenari
critici o testing.
In particolare, una security compliance evaluation è un processo che verifica la compliance (uniformità)
di un certo target a una o più politiche; vengono eseguiti tutti i controlli e produce un valore booleano per
le politiche, in base al valore booleano associato ad ogni controllo.

% 2- Tecnologie
\chapter{Tecnologie utilizzate}
In questo capitolo verranno descritte le attività preliminari per la realizzazione di questo progetto, le tecnologie utilizzate
unitamente alle motivazioni legate all'uso di questi sistemi rispetto ad altri. In particolare nel capitolo successivo verranno 
approfonditi a livello pratico i sitemi di raccomandazione Memory-based i quali sono stati utilizzati per l'implementazione della
soluzione. 

%% AGGIUNGERE RIFERIMENTI AI CAPITOLI SUCCESSIVI E INDICARE QUALI SONO LE PARTI CHE POI SONO STATE IMPLEMENTATE NELLA 
%% SOLUZIONE DI QUESTO PROGETTO

\section{Perché Python e Django}
\paragraph{Python}
Python è un linguaggio di programmazione ad alto livello, orientato agli oggetti, adatto, tra gli altri usi, a sviluppare applicazioni 
distribuite, scripting, computazione numerica e system testing; ideato e rilasciato pubblicamente per la prima volta nel 1991 dal suo 
creatore Guido van Rossum, programmatore olandese.

Python supporta diversi paradigmi di programmazione, come quello object-oriented (con supporto all'ereditarietà multipla), quello 
imperativo e quello funzionale, ed offre una tipizzazione dinamica forte. È fornito di una libreria built-in estremamente ricca, che 
unitamente alla gestione automatica della memoria e a robusti costrutti per la gestione delle eccezioni fa di Python uno dei linguaggi 
più ricchi e comodi da usare.

Inoltre è anche semplice da usare e imparare. Python, nelle intenzioni di Guido van Rossum, è nato per essere un linguaggio 
immediatamente intuibile. La sua sintassi è pulita e snella così come i suoi costrutti, decisamente chiari e non ambigui. I blocchi 
logici vengono costruiti semplicemente allineando le righe allo stesso modo, incrementando la leggibilità e l'uniformità del codice 
anche se vi lavorano diversi autori.

Un aspetto inusuale del Python è il metodo che usa per delimitare i blocchi di programma, che lo rende unico fra i linguaggi più diffusi.

\lstset{style=python_code_style}
\begin{lstlisting}[language=Python, caption={Esempio di programma in Python}]
# Testing if two strings are equals
def test(got, expected):
	if got == expected:
		prefix = ' OK '
	else:
		prefix = ' X '
	print('%s got: %s expected: %s' % (prefix, repr(got), repr(expected)))

def main():
	print('verbing')
	test('hail', 'hailing')
	test('swiming', 'swimingly')
	test('do', 'do')

if __name__ == '__main__':
	main()
\end{lstlisting}

Nei linguaggi derivati dall'ALGOL come Pascal, C e Perl, i blocchi di codice sono indicati con le parentesi oppure con parole chiave 
(il C ed il Perl usano { }; il Pascal usa begin ed end). In questi linguaggi è solo una convenzione degli sviluppatori il fatto di 
indentare il codice interno ad un blocco, per metterlo in evidenza rispetto al codice circostante. In Python invece di usare parentesi 
o parole chiave, usa l'indentazione stessa per indicare i blocchi nidificati in congiunzione col carattere "due punti" (:). Si può usare 
sia una tabulazione, sia un numero arbitrario di spazi, ma lo standard Python è di 4 spazi. 

Python è un linguaggio pseudocompilato: un interprete si occupa di analizzare il codice sorgente (semplici file testuali con 
estensione .py) e, se sintatticamente corretto, di eseguirlo e non esiste una fase di compilazione separata (come 
avviene in C, per esempio) che generi un file eseguibile partendo dal sorgente. \cite{python-documentation}

\paragraph{Django}
Django è un web framework di alto livello basato su Pyhton che permette di sviluppare rapitamente e con tutti i presupposti per un sistema sicuro, un sito web
perfettamente mantenibile, Django si occupa della maggiori grane del sviluppo web, così da permetterti di concentrarti sulla scrittura della tua app; inoltre 
è open-source e ha una comunità attiva, una documentazione completa e sempice da consultare.

Django aiuta a scrivere software con le seguenti caratteristiche \cite{django-documentation}:
\begin{description}
	\item[Versatile]: è usato per la creazione di praticamente tutti i tipi di siti web, a partire a sistemi per la gestioni di contenuti e wiki,
	social network e siti di notizie; può lavorare con qualunque client-side framework, gestire contenuti in quasi tutti i formati (inclusi HTML, 
	RSS feeds, JSON, XML, etc). Internamente permette la scelta e l'implementazione di qualsiasi funzionalità (es. molti dei database più popolari, etc.).
	\item[Sicuro]: aiuta gli sviluppatori a evitare gli errori più comuni in merito alla sicurezza provvedendo un framework costruito per eseguire 
	le operazioni in modo corretto e sicuro. Ad esempio, Django fornisce un modo sicuro per gestire gli account degli utenti e le relative
	password, evitando errori comuni come inserire informazioni riguardanti la sessione dell'utente nei cookies dove sarebbero vulnerabili (invece i cookie
	contengono soltanto una chiave, e i valori effettivi sono salvati nel database) o salvare direttamente una password invece di una hash password.
	\item[Mantenibile]: il codice di Django è scritto seguedo i principi e i pattern che incoraggiano la creazione di codice mantenibile e riusabile. Inoltre
	particolare, fa uso del principio "Don't Repeat Yourself" (DRY) così da ridurre al minimo le duplicazioni non necessarie, diminuendo la quantità di codice.
	Django raggruppa parti di codice legto in moduli seguendo le linee guida del pattern Model View Controller (MVC).
	\item[Portatile]: Django essendo scritto in Python, un linguaggio multi piattaforma, lo rende indipendente dal sistema operativo eseguito sul server, che
	sia Linux, Windows o Mac OS X. Per di più, Django è ben supportato da molti web hosting provider, che spesso provvedono a specifche infrastructure e 
	e documentazione per l'hosting di siti web in Django.
\end{description}


Un tradizionale sito web attende delle richieste HTTP dal browser web (o altri client). Quando viene ricevuta una richiesta, di tipo POST o GET, l'applicazione 
legge le informazioni contenute nel URL e altri possibili dati a seconda del tipo di richiesta. A seconda della richiesta è possibile che vengano letti o 
scritti dati da un database o altre operazioni che portino al soddisfacimento della richiesta. A quel punto l'applicazione ritorna una risposta al browser 
web, spesso in modo dinamico, creando una pagina HTML da mostrare in cui è possibile inserire o recuperare dati in placeholder in un template HTML.

\newpage

\begin{figure}[ht!]
    \centering
	\includegraphics[scale=0.3]{images/Django_doc.png}
	\caption{Schema generico di funzionamento di un applicativo web sviluppato con Django}
	\label{fig:django_doc}
\end{figure}

Un'applicazione web in Django tipicamente raggruppa il codice che gestisce questi diversi passaggi in file separati \cite{mdn-django-documentation}:
\begin{description}
	\item[URL]: mentre è possibile processare richieste da qualsiasi URL attraverso unas singola funzione, è più mantenibile scrivere diverse funzioni chiamate
	View per gestire ogni risorsa. un URL mapper è usato per reindirizzare le richieste HTTP alla view corretta in base all'URL della richiesta; inoltre è 
	possibile controllare se nell'URL è presente un particolare pattern di stringhe o numeri, e passare di conseguenza la richiesta alla funzione appropriata
	come dati da elaborare.
	\item[View]: una View è una funzione che gestisce le richieste HTTP, e restituisce una risposta HTTP. Le view accedono ai dati necessari per soddisfare la 
	richiesta attraverso i Model, e delegano la formattazione delle risposte ai template. 
	\item[Model]: i Model sono oggetti in Python che definiscono la struttura dei dati dell'applicazione, e provvedono meccanisci per gestirla (add, modify, 
	delete) e query per interpellare il database.
	\item[Template]: un template è un file di testo che definisce la struttura o il layout di un file (come una pagina HTML), attraverso placeholder per 
	rappresentare il contenuto effettivo. Una View può creare dinamicamente una pagina HTML usando un template HTML, popolandolo con dati presi dal Model. 
\end{description}

\section{Docker}
Docker è una piattaforma software che permette di creare, testare e distribuire applicazioni con la massima rapidità. Docker raccoglie 
il software in unità standardizzate chiamate \textit{container} che offrono tutto il necessario per la loro corretta esecuzione, incluse librerie, 
strumenti di sistema, codice e runtime. Con Docker, è possibile distribuire e ricalibrare le risorse per un'applicazione in qualsiasi ambiente, 
tenendo sempre sotto controllo il codice eseguito.

La tecnologia Docker utilizza solitamente il kernel di Linux e le sue funzionalità, come Cgroups e namespace, per isolare i processi in modo da poterli 
eseguire in maniera indipendente. Questa indipendenza è l'obiettivo dei container: la capacità di eseguire più processi e applicazioni in 
modo separato per sfruttare al meglio l'infrastruttura esistente pur conservando il livello di sicurezza che sarebbe garantito dalla 
presenza di sistemi separati.

\begin{figure}[ht!]
	\centering
	\includegraphics[scale=0.7]{images/Docker_Config_Container.png}
	\caption{Schematizzazione del contenuto di un Container in Docker}
	\label{fig:DCC}
\end{figure}

Gli strumenti per la creazione di container, come Docker, consentono il deployment a partire da un'\textit{immagine}, ciò semplifica la condivisione di 
un'applicazione o di un insieme di servizi, con tutte le loro dipendenze, nei vari ambienti.
Docker, considera i container come macchine virtuali modulari estremamente leggere, offrendo la flessibilità di creare, distribuire, 
copiare e spostare i container da un ambiente all'altro, ottimizzando così le app per il cloud.

I contenitori forniscono una modalità standard per impacchettare il codice della tua applicazione, le configurazioni e le dipendenze, in un oggetto singolo e 
condividono un sistema operativo installato sul server, operando come processi con risorse isolate, assicurando velocità, affidabilità e distribuzioni coerenti, 
indipendentemente dall’ambiente.

\section{Strutture dati gerarchiche}
Le tabelle di un database relazionale non sono gerarchiche (come nel XML), ma sono delle semplici liste piatte. I dati gerarchici sono 
constituiti da relazioni padre-figlio che non possono essere rappresentate in modo naturale nelle tabelle di questo tipo.
In questo caso, i dati gerarchici sono una collezione di informazioni dove ogni item ha un solo padre e nessuno o più figli
(ad eccezione del nodo radice che non ha un nodo padre); questo genere di rappresentazione delle informazioni, simili a un albero, 
può essere trovato in diversi ambiti di applicazione di un database, incluse discussioni su forum e mailing list, grafici di 
organizzazione di un business, categorie per gestire contenuti e categorie di prodotti. \\

\begin{figure}[ht!]
	\centering
	\includegraphics[scale=0.49]{images/MC_Rec_Tree.png}
	\caption{Esempio della rappresentazione gerarchica parziale dei dati nel progetto in questione}
	\label{fig:MC_Rec_Tree}
\end{figure}

%\newpage

Ci sono differenti modelli per poter gestire dati in modo gerarchico, i più importanti  presi in considerazione sono i seguenti:

\subsection{The Adjacency List Model}
Il primo approccio, e quello di più semplice implementazione, qui descritto è chiamato \textit{Adjacency List Model} o metodo ricorsivo;
è definito tale perchè il suo funzionamento si basa su una funzione che itera per tutto l'albero.\\
In questo modello, ogni item (nodo dell'albero) nella tablla contiene un puntatore al suo item padre; invece il nodo radice avrà un 
puntatore a un valore NULL per l'item padre.

La Tabella \ref{table:adjacency_list_model_table} è un esempio di possibile rappresentazione parziale dei dati nel database implementato 
in questo progetto secondo questo approccio, seguendo come riferimento la Figura \ref{fig:MC_Rec_Tree}.
 
\begin{table}[ht!]
\centering
\begin{tabular}[c]{| c | l | c |} 
	\hline
	id & name & parent \\ [0.5ex] 
	\hline
	\rowcolor{rootnodecell} 1 & Moon Cloud & NULL \\ [0.5ex] 
	\rowcolor{categorycell} 2 & Web & 1 \\ [0.5ex] 
	\rowcolor{categorycell} 3 & Protocol & 1 \\ [0.5ex] 
	\rowcolor{categorycell} 4 & Cloud & 1 \\ [0.5ex] 
	\rowcolor{evaluationcell} 5 & Portscan Report & 2 \\ [0.5ex] 
	\rowcolor{evaluationcell} 6 & Observatory Check & 2 \\ [0.5ex] 
	\rowcolor{evaluationcell} 7 & SSH Compliance Check & 3 \\ [0.5ex] 
	\rowcolor{evaluationcell} 8 & HTTPS Robustness Check & 3 \\ [0.5ex] 
	\rowcolor{categorycell} 9 & Azure & 4 \\ [0.5ex] 
	\rowcolor{categorycell} 10 & Aws & 4 \\ [0.5ex] 
	\rowcolor{categorycell} 11 & Dropbox & 4 \\ [0.5ex] 
	\rowcolor{evaluationcell} 12 & Sql Services Checker & 9 \\ [0.5ex] 
	\rowcolor{evaluationcell} 13 & Security Center Report & 9 \\ [0.5ex] 
	\rowcolor{evaluationcell} 14 & Macie Scanner & 10 \\ [0.5ex] 
	\rowcolor{evaluationcell} 15 & Sqs Checker & 10 \\ [0.5ex]
	\hline
\end{tabular}
\caption{Esempio di una possibile tabella per gestire dati in modo gerarchico secondo l'Adjacency List Model}
\label{table:adjacency_list_model_table}
\end{table}

Il vantaggio di usare questo modello sta nella sua semplicità di costruzione, sopratutto a livello di codice client-side, 
e di restituzione dei figli di un nodo. Mentre diventa problematico se si lavora in puro codice SQL e nella maggior parte dei linguaggi 
di programmazione, è lento e poco efficente, perchè è necessaria una query per ogni nodo dell'albero, e visto che ogni query impiega 
un certo periodo di tempo, questo rende la funzione molto lente quando si lavora con alberi di grandi dimensioni, inoltre molti linguaggi 
non sono ottimizzati per funzioni ricorsive. Per ogni nodo, la funzione crea una nuova istanza di se stessa e ogni istanza occupa 
una porzione di memoria e impiega un certo tempo per inizializzarsi, più grande è l'albero e più questo processo sarà portato a termine 
in maggior tempo.

\subsection{The Nested Set Model}
Il secondo approccio analizzato è il \textit{Nested Set Model}, che permette di osservare la gerarchia in un modo diverso, non 
come nodi e linee (come se fosse un albero), ma come container innestati. \\

\begin{figure}[ht!]
	\includegraphics[scale=0.30]{images/MC_Rec_NSM_Container.png}
	\caption{Esempio della gestione di dati in modo gerarchico secondo il Nested Set Model, utilizzando dati presi dal database del 
	progetto in questione}
	\label{fig:MC_Rec_NSM_Container}
\end{figure}

Con questo sistema la gerarchia viene mantenuta, secondo il principio cui un nodo padre contiene e suoi figli e questa forma di 
gerarchia viene mantenuta in tabella attraverso l'uso di due attributi aggiuntivi come è possibile osservare dalla Tabella 
\ref{table:nested_set_model_table} seguente. 

\begin{table}[ht!]
\centering
\begin{tabular}[c]{| c | l | c | c |} 
	\hline
	id & name & lft & rght \\ [0.5ex] 
	\hline
	\rowcolor{rootnodecell} 1 & Moon Cloud & 1 & 100 \\ [0.5ex] 
	\rowcolor{categorycell} 2 & Web & 86 & 99 \\ [0.5ex] 
	\rowcolor{categorycell} 3 & Protocol & 80 & 85 \\ [0.5ex] 
	\rowcolor{categorycell} 4 & Cloud & 4 & 29 \\ [0.5ex] 
	\rowcolor{evaluationcell} 5 & Portscan Report & 91 & 92 \\ [0.5ex] 
	\rowcolor{evaluationcell} 6 & Observatory Check & 89 & 90 \\ [0.5ex] 
	\rowcolor{evaluationcell} 7 & SSH Compliance Check & 83 & 84 \\ [0.5ex] 
	\rowcolor{evaluationcell} 8 & HTTPS Robustness Check & 81 & 82 \\ [0.5ex] 
	\rowcolor{categorycell} 9 & Azure & 13 & 26 \\ [0.5ex] 
	\rowcolor{categorycell} 10 & Aws & 5 & 12 \\ [0.5ex] 
	\rowcolor{categorycell} 11 & Dropbox & 27 & 28 \\ [0.5ex] 
	\rowcolor{evaluationcell} 12 & Sql Services Checker & 22 & 23 \\ [0.5ex] 
	\rowcolor{evaluationcell} 13 & Security Center Report & 20 & 21 \\ [0.5ex] 
	\rowcolor{evaluationcell} 14 & Macie Scanner & 8 & 9 \\ [0.5ex] 
	\rowcolor{evaluationcell} 15 & Sqs Checker & 10 & 11 \\ [0.5ex]
	\hline
\end{tabular}
\caption{Esempio di una tabella per gestire dati in modo gerarchico secondo il Nested Set Model}
\label{table:nested_set_model_table}
\end{table}

Come è possibile osservare dalla Tabella \ref{table:nested_set_model_table} la gerarchia dei dati viene rappresentata attraverso l'uso 
degli attributi 'left' e 'right' per rappresentare l'annidamento dei nodi (il nome delle colonne: 'left' e 'right', hanno significati 
speciali in SQL; per questo motivo si identificano questi campi con i nomi 'lft' e 'rght'). \\

\begin{figure}[ht!]
	\includegraphics[scale=0.40]{images/MC_Rec_NSM_Tree.png}
	\caption{Esempio della gestione di dati in modo gerarchico secondo il Nested Set Model, utilizzando dati presi dal database del 
	progetto in questione}
	\label{fig:MC_Rec_NSM_Tree}
\end{figure}

L'assegnazione di questi valori ad ogni nodo viene effetuata seguendo questo procedimento: ogni nodo dell'albero viene visitato due volte, 
assegnando i valori in ordine di visita, e in entrambe le visite. Quindi vengono associati ad ogni nodo due numeri, memorizzati come 
due attributi. 
Più precisamente si inizia la visita dell'albero partendo da sinistra e continuando verso destra, un livello alla volta, scendendo per ogni
nodo i suoi figli, assegnando i valori al campo left, prima di assegnare un valore al campo right, e successivamente si continua verso 
destra. Questo approccio è chiamato \textit{Modified preorder tree traversal algorithm} (MPTT). A partire da questa tecnica è stato 
ideata la struttura della tassonomia delle evaluation e dei controlli implementate nella soluzione proposta in questa tesi, con 
l'ausilio di un package di Python chiamato MPTT.

Più semplicemente se si osserva la parte superiore della Figura \ref{fig:MC_Rec_NSM_Container} possiamo notare che la numerazione dei nodi, viene
effettuata a partire da container più esterno da sinistra e continua verso destra.

A prima vista questo approccio può sembrare più complicato da comprendere rispetto all'Adjacency List Model, ma quest'utlimo è
molto più veloce quando si vuole recuperare i nodi, visto che basta una query, mentre è più lento per operazioni di aggiornamento e 
cancellazione dei nodi; in quest ultimo il grado di complicatezza dell'operazione è determinato dal nodo che si vuole cancellare, a 
partire dal caso più semplice, il nodo foglia (un nodo senza figli) fino al caso più complicato, quando si vuole cancellare il nodo 
radice.

\newpage

\section{Sistemi di raccomandazione}
Un sistema di raccomandazione (\textit{Recommendation System}) è un sistema che consiglia ad utenti uno o più item esistenti 
in un database. L'\textit{item} è un qualsiasi cosa di interesse agli utenti, come prodotti, libri o giornali. Quando si eseguire
una raccomandazione si hanno delle aspettative che l'item raccomandato possa essere tra quelli di maggiore interesse; in altre parole, 
devono essere in accordo con i gusti degli utenti. 
\\
\\
Oggigiorno si possono trovare due principali trend di sistemi di raccomandazione: 
\begin{description}
	\item[Content-based filtering](CBF): un item viene raccomandato ad un utente se esso è simile ad altri item di interesse o piaciuti
	in passato, vengo presi in considerazione prima gli item con alte valutazioni o quelli molto utilizzati. Ogni item ha associate
	delle informazioni che lo descrivono, questo insieme di dati viene definito metadati, e sono fondamentali per il processo di 
	raccomandazione. 
	\item[Collaborative filtering](CF): un item viene raccomandato ad un utente se i suoi vicini (altri utenti simili) sono 
	interessati a quello stesso item.   
\end{description}

\begin{figure}[ht!]
	\centering
	\includegraphics[scale=0.5]{images/recommender_systems.png}
	\caption{Categorizzazione generale dei sistemi di raccomandazione}
	\label{fig:recommender_systems}
\end{figure}

Entrambi gli approcci (CBF e CF) hanno i loro punti di forza e di debolezza. Il primo algoritmo si focalizza sul contenuto degli item e
sugli interessi del singolo utente e propone item differenti a utenti differenti, questo significa che ogni utente può ricevere 
raccomandazioni uniche; e questo è un vantaggio. 
Tuttavia la più grande limitazione del CBF è il fatto di non poter determinare se un utente è interessato ad un item in modo implicito,
perchè analizza solamente direttamente i metadati del prodotto e non considera gli interessi di altri utenti, i quali potrebbero 
suggerire item che non verrebero notati con questo approccio.
Per quanto riguarda il CF, nel caso siano presenti molti contenuti e proprietà associati agli item allora il sistema CF consuma molte 
risorse e tempo per poter analizzarli, nel contempo a questo algoritmo non interessano queste informazioni. Una raccomandazione 
viene fatta sulla base delle valutazioni degli utenti per gli item, o sugli usi che gli utenti fanno degli item e questo è il suo punto 
di forza perchè non si trova a dover analizzare item ricchi di informazioni. Allo stesso tempo è anche il suo punto debole, perchè può
portare a suggerimenti che potrebbero essere considerati poco adatti sulla base della poca relazione con i profili di alcuni utenti. 
Questo problema è accentuato quando sono presenti nel database molti item che non hanno valutazioni o non sono stati mai usati dagli 
utenti.
\cite{model-based-approach-for-collaborative-filtering}

Un sistema di raccomandazione filtra i dati usando differenti algoritmi e raccomanda gli item più rilevanti agli utenti, attraverso 
un procedimento a 3 fasi:

\begin{description}
	\item[Raccolta di dati]: questo è il primo step e anche quello più importante per poter costruire un sistema di 
	raccomandazione che produca risultanti rilevanti e consistenti. I dati possono essere raccolti in due modi: esplicitamente,
	cioè attraverso la raccolta diretta di informazioni fornite dagli utenti, ad esempio le valutazioni di un prodotto; mentre 
	attarverso l'approccio implicito, vengono raccolti dati che non sono prodotti in modo intenzionale dall'utente ma ottenuti
	dai costanti flussi di dati come la cronologia di ricerca, i click effettuati, lo storico degli ordini, etc.
	\item[Memorizzazione di dati]: la quantità di dati definisce quanto efficace un modello di raccomandazione possa di
	diventare. Ad esempio, in un sistema di raccomanzione per film, maggiori sono le valutazioni fornite dagli utenti, e 
	migliore sarà il sistema di raccomandazione per gli altri utenti. Il tipo di dati che si vuole raccogliere determina
	anche il supporto di memorizzazione più adatto.   
	\item[Filtraggio dei dati]: dopo la fase di raccolta e memorizzazione dei dati, essi vanno filtrati per poter estrarre
	le informazioni rilevanti per poter effettuare le raccomandazioni finali, e sono già disponibili diversi algoritmi che
	semplificano quest'ultima fase. 
\end{description}

I sistemi di raccomandazione possono essere suddivisi nelle seguenti categorie, ma spesso si preferisco degli approcci ibridi, delle
combinazioni di sistemi di raccomandazione basati sul Contenuto (\textit{Content-based filtering}) e di
quelli Collaborativi (\textit{Collaborative filtering}) in modo da essere più efficaci e sfruttare i pregi di entrambi gli approcci.


\subsection{Content-based filtering}
Un Content-based filtering è un sistema di raccomandazione in cui vengono suggeriti, rispetto ad un item (oggetti o prodotti), quelli 
più simili, il confronto viene effettuato sfruttand i metadati, come il genere, una descrizione, uno o più autori, la categoria di 
appartenenza, etc.; l'idea base che si trova dietro questi sistemi, si basa sul fatto che se ad un utente piace o interessa un particolare 
item allora gli piaceranno anche altri con caratteristiche o proprietà simili.

Questo algoritmo suggerisce prodotti che piacevano all'utente nel passato ed è limitato a item dello stesso tipo. Un 
Content-based recommender fa riferimento a quegli approcci, che provvedono raccomandazioni comparano la rappresentazione del
contenuto che descrive un item e la rappresentazione del contenuto dell'item interessato dall'utente. 

Questi metodi sono usati quando si conoscono a priori i metadati sugli item che si vuole suggerire, ma nulla sugli utenti.
In questo sistema, delle \textit{keyword} sono utilizzate per caratterizzare gli item e un profilo dell'utente è 
costruito per memorizzare quali item sono di suo interesse. In altre parole, questi algoritmi cercano di raccomandare quello che 
l'utente ha valutato positivamente o usato nel passato e sta esaminando nel presente. La costruzione del profilo dell'utente,
spesso temporaneo, non viene basata su un modulo di registrazione che l'utente stesso deve compilare, ma su informazioni
lasciate indirettamente dall'utente, le quali possono essere: i prodotti che ha maggiormente cercato e acquistato, quelli che sono 
stati inseriti nella lista dei desideri, etc.. Più precisamente, tra vari item candidati da raccomandare all'utente si passa per un 
processo di confronto con gli item piaciuti dall'utente e gli item migliori vengono suggeriti.


\subsection{Collaborative filtering}
I filtri Collaborativi (\textit{Collaborative filtering}) per poter funzionare come prima cosa costruiscono un database di preferenze 
degli utenti sulla base di un insieme di item (o prodotti), che a loro volta possono essere presenti un database,
sfruttano tecniche di analisi dei dati per risolvere il problema di aiutare gli utenti a trovare gli item che gli potrebbero piacere 
producendo, eventualmente una lista dei top-N item da raccomandare per un dato utente.
Un utente è sottoposto ad un processo di matching all'interno del database per scoprire quali sono i possibili \textit{neighbors},
che corrispondono ai possibili utenti aventi storicamente delle preferenze in comune al lui. A questo punto gli item maggiormente 
preferiti dai neighbors sono raccomandati all'utente visto che potrebbero essere di suo interesse. 

Questi sistemi tentano di predirre la valutazione o la preferenza che un utente darebbe a un item basandosi su preferenze date da altri 
utenti, queste ultime possono essere ottenute o in modo esplicito dagli utenti o tramite misurazioni implicite. 
Inoltre i filtri Collaborativi non richiedono l'uso di metadati associati agli item come nella loro controparte, i filtri Content-based. 

Tuttavia, restano ancora oggi alcune sfide significative a cui sono sottoposti i sistemi di raccomandazione basati su 
filtraggio Collaborativo.
\begin{description}
	\item Il primo obbiettivo è quello di migliorare la scalabilità degli algoritmi di filtri Collaborativi; questi algoritmi sono in grado di cercare
	anche diecimila di potenziali neighbors (utenti simili) in tempo reale, ma la richiesta dei sistemi moderni è di cercare dieci milioni di 
	potenziali neighbors, per questo motivo possono nascere problemi di performance con i singoli utenti quando essi hanno molte informazioni.
	\item Il secondo obbiettivo è quello di migliorare la qualità dei sistemi di raccomandazione per gli utenti. Gli utenti vogliono
	raccomandazioni di cui possono fidarsi e che possono aiutarli a trovare item che potrebbero essere di loro gusto.
\end{description}
Per certi versi questi due obbiettivi sono in conflitto tra di loro e per ottenere dei risultati validi e di una certa importanza è 
necessario trattarli in contemporanea perchè aumentare solamente la scalabilità diminuirebbe la sua qualità e viceversa. 
\cite{item-based-collaborative-filtering} 

Il principale modello di filtro collaborativo studiato in questo elaborato è il metodo definito come \textit{Memory-based} e il 
vantaggio di utilizzare queste tecniche sta nel fatto di essere semplici da implementare e i risultati ottenuti sono altrettanto 
semplici da spiegare; mentre ci possono essere anche filtri collaborativi che sfruttano metodi \textit{Model-based} che si basano sulla 
fattorizzazione di matrici e sono molto più funzionali per gestire il problema della sparsità dei dati. Questi ultimi sono sviluppati
usando algoritmi di data mining e machine learning per predirre le valutazioni di utenti su item senza valutazioni, inoltre sono spesso 
associati a tecniche come la dimensionality reduction per migliorare la precisione.


\paragraph{Model-based filtering} 
Gli algoritmi Model-based tentano di comprimere grandi database in un modello ed effettuare il processo di raccomandazione applicando dei
meccanismi di riferimento all'interno di questo modello. 
I CF Model-based posssono rispondere alle richieste degli utenti istantaneamente. \cite{model-based-approach-for-collaborative-filtering}


\subsection{Cold Start problem} 
Cosa succederebbe se un nuovo utente o un nuovo item è aggiunto al dataset? Questa situazione è chiamata \textit{Cold Start}. Ci possono
essere due tipi di Cold Start:
\begin{description}
	\item[Visitor Cold Start]: si verifica quando un nuovo utente è stato aggiunto al database, visto che non c'è alcuno storico relativo ad esso, il sistema non
	sa le le sue preferenze; per questo motivo diventa molto più difficile raccomandare prodotti a quel particolare utente. Per risolvere questo problema,
	a livello teorico, si potrebbe applicare un procedimento di raccomandazione basata sulla popolarità dei prodotti, ma solo una volta che si è venuti a 
	conoscenza delle preferenze dell'utente, sarà possibile generare delle raccomandazioni più precise e adeguate alle sue esigenze.
	\item[Item Cold Start]: si verifica quando un nuovo item viene inserito nel sistema. L'azione dell'utente è quella più importante per determinare
	il valore di questo item; maggiore l'interazione un item riceve, più è facile che venga raccomandato all'utente giusto.  
\end{description}





% 3- Recommendation systems
\chapter{Collaborative filtering}
\label{chp:03-recommendationSystems}
In questo capitolo verranno approfonditi gli algoritmi di raccomandazione implementati nella soluzione, mostrando le porzioni di 
codice e spiegando i vari passagi che portano ad ottenere delle raccomandazioni.
% - nel capitolo dei recommendation system parlare dei collaborative filter in modo più dettagliato e legato al progetto -


\section{Memory-based} 
I Filtri Collaborativi Memory-based sono stati introdotti per via delle osservazioni che vennero fatte dalla comunità, dicenddo che
gli utenti si fidano maggiormente delle raccomandazioni di altri che la pensano allo stesso modo. Questi metodi mirano a calcolare 
le relazioni tra utenti e item attraverso lo schema dei vicini che identifica sia coppie di item che tendono ad essere usati insieme 
o hanno un grado di similarità alto o utenti con uno storico di item usati simile. \cite{taxonomy-of-recommender-agents-on-the-internet}
Questi approcci divennero molto famosi grazie alla loro semplicità di implementazione, molto intuitivi, non necessitano di operazioni
di training sui dati e regolazione di molti parametri, inoltre l'utente può capire la ragione che sta dietro ogni raccomandazione. 

Questa categoria di sistemi di raccomandazione sono definiti anche \textit{Neighborhood-based} e possono essere ulteriormente classificati 
in due sottocategorie:


\subsection{User-based filtering} 
Questo sistema, definiti anche con l'acronimo UB-CF (\textit{User-based Collaborative Filter}) basa tutto il suo funzionamento sulla 
comunità di utenti, maggiore è la sua dimensione e l'attività degli utenti con item o servizi e migliori potranno essere le 
raccomandazioni. Questo algoritmo fornisce dei suggerimenti ad un utente sulla base di uno o più vicini, e la similarità tra utenti può
essere determinata sulla base degli item che l'utente ha utilizzato o valutato.

Molti di questi approcci possono essere generalizzati dall'algoritmo organizzato nei seguenti passi:
\begin{enumerate}
	\item Specificare qual'è l'utente a cui si vuole applicare l'algoritmo di raccomandazione e recuperare quali utenti possono 
	avere dato valutazioni o usato item simili al utente target. Piuttosto che recuperare tutti gli utenti, per velocizzare l'esecuzione
	dell'algoritmo, è possibile selezionare soltanto un gruppo di utenti in modo casuale oppure associare dei valori di similarità tra 
	tutti gli utenti e confrontando questi valori con quello dell'utente target, selezionare i relativi utenti che superano una soglia
	scelta, oppure utilizzare tecniche di clustering.
	\item Estrarre quegli item a cui l'utente target non ha mai interagito e per questo motivo gli possono interessare, e mostrarli 
	sottoforma di raccomandazioni.
\end{enumerate}

\begin{figure}[ht!]
	\centering
	\includegraphics[scale=0.5]{images/UB_CF_ex.png}
	\caption{Esempio di applicazione di un sistema di raccomandazione User-based}
	\label{fig:UB_CF}
\end{figure}

Questi approcci sono facilmente implementabili, indipendenti dal contesto in cui sono applicati e possono essere più accurati rispetto
a tecniche basate sul Content-based filtering; dall'altra parte all'aumentare del numero di utenti che vado a considerare per fare le 
raccomandazioni migliore è la precisione di questo processo ma anche è maggiore il costo in termini di tempo.  

Nella soluzione proposta in questa tesi, l'algoritmo UB-CF viene implentato come funzione che prende in ingresso un parametro 
(\textit{user\_other\_id}), come è possibile osservare da \ref{lst:UB_CF_1} corrispondente all'identificativo per l'utente, 
e restituisce una lista di raccomandazioni (\textit{similar\_user\_evaluations}) corrispondenti alle Evaluation simili a quelle usate da altri utenti. 

\lstset{style=python_code_style}
\begin{lstlisting}[language=Python, label=lst:UB_CF_1]
# User recommendation algortihm
def user_recommendation_alg(user_other_id):
\end{lstlisting}

Più precisamente come funziona l'algoritmo si suddivide nei seguenti passi:
\begin{description}
	\item il primo passo è quello di recuperare sulla base del parametro in ingresso alla funzione, lo user\_other\_id,
	tutte le evaluation che l'utente in questione ha utilizzato;
	\begin{lstlisting}[language=Python, label=lst:UB_CF_2]
		# Select the target user and its evaluations
		target_user_evaluations = User.objects.get(other_id=user_other_id).evaluations.all()\
																					.values('other_id', 'id', 'parent_id')\
																					.order_by('other_id')
	\end{lstlisting}
	\item il secondo passo consiste nel selezionare le Evaluation usate dagli altri utenti, e viene create una lista di queste Evaluation;
	\begin{lstlisting}[language=Python, label=lst:UB_CF_3]
		# Select all other users and theirs evaluations
		other_users = User.objects.exclude(other_id=user_other_id)
		# Creating a list with all the evaluations of other users
		other_users_evaluations = []
		for o_users_evaluation in other_users:
			for evaluation in o_users_evaluation.evaluations.all().values('other_id', 'id', 'parent_id')\
																	.order_by('other_id'):
				other_users_evaluations.append(evaluation)
	\end{lstlisting}
	\item il terzo passo consiste nell'andare a determinare quali tra le Evaluation dell'utente a cui si vuole raccomandare quali sono quelle 
	simili usate dagli altri utenti. Per determinare le Evaluation simili si è andato a confrontare il paramentro \textit{parent\_id} (identifica
	all'interno della tassonomia quale sia il nodo padre per quella Evaluation), associato ad ogni Evaluation, in questo modo si è andati a selezionare
	soltanto gli item appartenenti a una stessa categoria, eliminando eventuali nodi duplicati. E componendo una lista finale con le Evaluation
	restanti.
	\begin{lstlisting}[language=Python, label=lst:UB_CF_4]
		# Comparing target user's evaluations and other user's evaluations, and if there is a match the evaluation is
		# added to the 'similar_evaluations' list (the matching is made comparing the 'parent_id')
		similar_user_evaluations = []
		for t_user_evaluation in target_user_evaluations:
			for o_users_evaluation in other_users_evaluations:
				# Taking only the evaluations that have: different other_id (excluding the target evaluation
				# in the recommendation) and same parent_id and the evaluations that weren't added to 'target_user_evaluations'
				# list and to 'similar_user_evaluations'
				if ((t_user_evaluation['other_id'] != o_users_evaluation['other_id']) and  # Evaluations must have different 'other_id'
						(t_user_evaluation['parent_id'] == o_users_evaluation['parent_id']) and  # Evaluations must have the same 'parent_id'
						# Evaluation in all_other_evals list mustn't be already added to \
						not (o_users_evaluation in target_user_evaluations) and  # the 'target_user_evaluations' list or
						not (o_users_evaluation in similar_user_evaluations)):  # the 'similar_user_evaluations' list
					similar_user_evaluations.append(o_users_evaluation)
	\end{lstlisting}
	
\end{description}

%% ESEMPIO DI UNA CHIAMATA CON VALORE DI RITORNO
%\begin{figure}[ht!]
%	\centering
%	\includegraphics[scale=0.5]{images/UB_CF_test.png}
%	\caption{Esempio di esempio di una risposta in JSON a una chiamata Rest all'algoritmo di raccomandazione UB-CF.}
%	\label{fig:UB_CF_resp_json}
%\end{figure}
\ \\
Nel capitolo successivo verrano mostrati degli esempio pratici in cui è stato applicato questo algoritmo.

\newpage

\subsection{Item-based filtering} 
Quando viene applicato per milioni di utenti e item, l'algoritmo UB-CF non è molto efficente, per via della complessa computazione della 
ricerca di utenti simili; così in alternativa è stato introdotto l'algoritmo di filtraggio Item-based, definito anche IB-CF 
(\textit{Item-based Collaborative Filter}) dove piuttosto che effetuare il confronto tra utenti simili, viene fatto un confronto tra 
gli item dell'utente a cui si vuole raccomandare e i possibili item simili.

\begin{figure}[ht!]
	\centering
	\includegraphics[scale=0.5]{images/IB_CF_ex.png}
	\caption{Esempio di applicazione di un sistema di raccomandazione IB-CF.}
	\label{fig:IB_CF}
\end{figure}
\ \\
Questi sistemi sono estremamente simili ai sistemi di raccomandazione Content-based, e identificano item simili in base a come utenti gli
hanno usati nel passato \cite{item-based-collaborative-filtering}.


%% CODICE 
\lstset{style=python_code_style}
\label{lst:CF_IB_Evaluation}
\begin{lstlisting}[language=Python, caption={Implementazione del CF-IB per le Evaluation presenti in Moon Cloud.}]
# Item recommendation algortihm
def item_recommendation_alg(item_other_id):
	"""
	For a 'target' evaluation this algorithm suggest other evaluations that belong to the same category (this means
	that they have the same 'parent_id').
	:param item_other_id: value representing the other_id of the evaluation.
	:return: a list of evaluations.
	"""
	# Selecting the evaluation, which is applied this algorithm, from its other_id
	# SELECT * FROM recommendation_app_evaluation WHERE other_id = %(item_other_id)s AND node_type = 'eva'
	target_eval = \
		Evaluation.objects.filter(Q(other_id=item_other_id) & Q(node_type="eva"))\
							.values('other_id', 'id', 'parent_id')[0]

	# Selecting the other evaluations, excluding the target evaluation
	# SELECT * FROM recommendation_app_evaluation WHERE other_id != %(item_other_id)s AND node_type = 'eva'
	all_other_evals = Evaluation.objects.filter(~Q(other_id=item_other_id) & Q(node_type="eva"))\
										.values('other_id', 'id', 'parent_id').order_by('other_id')

	# Creating a list with all the evaluations that are similar to the target evaluation (comparing the parent_id)
	similar_item_evaluations = []
	for evaluation in all_other_evals:
		# Taking only the evaluations that have: different other_id (excluding the target evaluation
		# in the recommendation) and same parent_id and the evaluations that weren't added to similar_item_evaluations
		# list
		if ((target_eval['other_id'] != evaluation['other_id']) and  # Evaluations must have different 'other_id'
				(target_eval['parent_id'] == evaluation['parent_id']) and  # Evaluations must have same 'parent_id'
				# Evaluation in all_other_evals list mustn't be already added to \
				not (evaluation in similar_item_evaluations)): # the 'similar_item_evaluations' list
			similar_item_evaluations.append(evaluation)

	return similar_item_evaluations	
\end{lstlisting}

%% ESEMPIO DI UNA CHIAMATA
\begin{figure}[ht!]
	\centering
	\includegraphics[scale=0.5]{images/IB_CF_Evaluation_test.png}
	\caption{Esempio di esempio di una risposta in JSON a una chiamata Rest all'algoritmo di raccomandazione IB-CF per le Evaluation
	supportate da Moon Cloud.}
	\label{fig:IB_CF_Eval_resp_json}
\end{figure}


\lstset{style=python_code_style}
\label{lst:IB_CF_Target}
\begin{lstlisting}[language=Python, caption={Implementazione del IB-CF per i Target supportati da Moon Cloud.}]
def target_recommendation_alg(target_id):
	"""
	For a target chose by a user this algorithm search the possible evaluations that can be recommended for that user.
	:param target_id: identifier of a particular target.
	:return: a list of evaluations.
	"""
	# Retriving all the evaluations in the database
	evaluations = Evaluation.objects.filter(node_type="eva")

	# Saving in the target_evaluations list the evaluations which controls have target_type_id equal to target_id
	target_evaluations = []
	for evaluation in evaluations: # Scanning all the evaluations
		for evaluation_controls in evaluation.controls.filter(target_type_id=target_id):
			if not(evaluation in target_evaluations): # Excluding evaluations duplicated
				target_evaluations.append(evaluation)

	# Converting the Evaluation model's instance in a dict and putting the evaluation, as a dict, in a list
	target_evaluations_serializer = EvaluationSerializer(target_evaluations, many=True)

	return target_evaluations_serializer.data
\end{lstlisting}

%% ESEMPIO DI UNA CHIAMATA
\begin{figure}[ht!]
	\centering
	\includegraphics[scale=0.5]{images/IB_CF_Target_test.png}
	\caption{Esempio di esempio di una risposta in JSON a una chiamata Rest all'algoritmo di raccomandazione IB-CF per i 
	Target supportarti da Moon Cloud.}
	\label{fig:IB_CF_Target_resp_json}
\end{figure}


\subsection{Hybrid filtering} 

%% CODICE
\lstset{style=python_code_style}
\label{lst:CF_Hybrid}
\begin{lstlisting}[language=Python, caption={Implementazione di un sistema raccomandazione ibrido
	che metta insieme le Raccomandazioni generate dagli algoritmi Item-based e User-based per le Evaluation presenti in Moon Cloud.}]
# HYBRID IMPLEMENTATION OF USER RECOMMENDATION ALGORTIHM AND ITEM RECOMMENDATION ALGORITHM
@api_view(['GET'])
def hybrid_recommendation(request, user_other_id):
	"""
	For a User-X this algorithm compare his evaluations with other users's evaluations and return all the evaluations
	that are similiar (same 'parent_id') to User-X's evaluations; then from all the evaluations used by the User-X other
	recommendation are computed using the item recommendation algorithm; and all the recommendations from the first step
	(here is used the user_recommendation_alg) and the second step (here is used the item_recommendation_alg)
	are put together.
	:param request: http GET request.
	:param user_other_id: value representing the other_id of a User.
	:return: json response with the evaluations to recommend.
	"""

	# Trying to retrive the actual User with user_other_id
	user = User.objects.get(other_id=user_other_id)

	# Taking from the user_recommendation_alg the evaluation recommended from this approach (similar_user_evaluations)
	# and the user's evaluations (target_user_evaluations)
	target_user_evaluations, similar_user_evaluations = user_recommendation_alg(user_other_id)

	# For every evaluation used by users is extracted all other possible evaluations that have the same 'parent_id'
	similar_item_evaluations = []
	for t_user_evaluation in target_user_evaluations:  # for every target user's evaluations
		for item_evaluation in item_recommendation_alg(t_user_evaluation['other_id']):  # is applied the item_recommendation algorithm
			# Taking only the evaluations that have: different other_id (excluding the target evaluation
			# in the recommendation) and same parent_id and the evaluations that weren't added to 'similar_item_evaluations'
			# list or to 'similar_user_evaluations' or to 'target_user_evaluations'
			if ((t_user_evaluation['other_id'] != item_evaluation['other_id']) and # Evaluations must have different 'id'
					(t_user_evaluation['parent_id'] == item_evaluation['parent_id']) and # Evaluations must have the same 'parent_id'
					# Evaluation in all_other_evals list mustn't be already added to \
					not (item_evaluation in similar_item_evaluations) and # the 'similar_item_evaluations' list,
					not (item_evaluation in similar_user_evaluations) and # the 'similar_user_evaluations' list or
					not (item_evaluation in target_user_evaluations)): # the 'target_user_evaluations' list
				similar_item_evaluations.append(item_evaluation)

	# Putting together the evaluations recommended in similar_user_evaluations list and similar_item_evaluations list
	similar_evaluations = []
	# Adding to similar_evaluations list the evaluation in the similar_user_evaluations list
	for s_user_evaluation in similar_user_evaluations:
		similar_evaluations.append(s_user_evaluation)
	# Adding to similar_evaluations list the evaluation in the similar_item_evaluations list
	for item_evaluation in similar_item_evaluations:
		# Taking only the evaluations that weren't added to \
		if (not (item_evaluation in similar_evaluations) and # the 'similar_evaluations' list or
				not (item_evaluation in target_user_evaluations)): # the 'target_user_evaluations' list
			similar_evaluations.append(item_evaluation)
	similar_evaluations = sorted(similar_evaluations, key=lambda i: i['other_id'])

	return JsonResponse(similar_evaluations, safe=False)
\end{lstlisting}

%% ESEMPIO DI UNA CHIAMATA
\begin{figure}[ht!]
	\centering
	\includegraphics[scale=0.5]{images/CF_Hybrid_test.png}
	\caption{Esempio di esempio di una risposta in JSON a una chiamata Rest all'algoritmo di raccomandazione Ibrido.}
	\label{fig:CF_Hybrid_resp_json}
\end{figure}

% 4- Descrizione del mio progetto
\chapter{Descrizione della soluzione}\label{chp:04-solution}
In questo capitolo è approfondito l'aspetto puramente pratico e le fasi che hanno portato alla realizzazione della soluzione; 
inoltre vengono mostrate le applicazioni pratiche degli aspetti teorici enunciati nei capitoli precedenti.
%
\vspace{1.5 cm}
\hfill\break
% BREVE DESCRIZIONE DELLA SOLUZIONE
La soluzione proposta in questa testi implementa un servizio di API REST, che si appoggia a un database Postgres, accessibile 
attraverso apposite URL; questo servizio permette di effettuare richieste al sistema di raccomandazione e di effettuare degli 
aggiornamenti a questa base di dati.
%
\section*{Preparazione della base di dati}
Prima di poter costruire il sistema di raccomandazione proposto in questa tesi, sono state eseguite delle operazioni preliminari 
per poter impostare il progetto di Django e la relativa applicazione che implementerà effettivamente la soluzione. 
% TASSONOMIE E DATABASE CON I MODELS (package MPTT)
Come descritto nei capitoli precedenti per procedere alla costruzione di un sistema di raccomandazione bisogna avere a disposizione 
una base di dati solida da cui attingere tutte le informazioni; ed è proprio questo il primo passo che è stato seguito, disegnare 
e progettare un database da cui partire per la realizzazione degli algoritmi proposti.
In generale Moon Cloud possiede una struttura delle Evaluation e dei Control ad albero, di conseguenza anche le tabelle del database 
rispecchiano questa struttura, partendo dalle considerazioni fatte sulle tecniche descritte nei capitoli precedenti si è deciso, che 
per implementare un database relazionale che gestisse dati gerarchici, di utilizzare la tecnica
definita come \textit{Modified Preorder Tree Traversal Algorithm}, la quale permette di massimizzare l'efficienza nelle 
operazioni di recupero dei dati, e quindi velocizzare i processi di raccomandazione, ma scendendo a compromessi per quanto riguarda le 
operazioni d'inserimento e spostamento dei nodi all'interno della struttura. 
Il package MPTT è una app di Django che ha come obiettivo quello di semplificare il più possibile la realizzazione dei Model, utilizzati 
per la generazione della base di dati, e la gestione della struttura di dati ad albero; si prende cura di tutti i dettagli riguardanti la 
realizzazione delle tabelle e dei campi \textit{left} e \textit{right} associati ad ogni nodo della tassonomia, mettendo a disposizione dei 
tool per poter lavorare con le istanze dei Model. Nel Listing \ref{lst:model} è possibile trovare le porzioni principali del 
codice costituenti i Model.
%
% MODELS CODES
\lstset{style=python_code_style}
\begin{lstlisting}[language=Python, label=lst:model, caption={Parti principali del codice che costituiscono i Model della soluzione.}]
# TARGET TYPE MODEL
 
class TargetType(models.Model):
    """
    Target supported by Moon Cloud system
    """
    TYPES = (
        ('host', 'host'),
        ('windows', 'windows'),
        ('url', 'url'),
        ('azure', 'azure'),
        ('aws', 'aws')
    )
    name = models.CharField(max_length=150, choices=TYPES, default="host")
    descr = models.TextField(max_length=1000, default="none")  # Description of a target
 
    def __str__(self):
        return str(self.name)
 
    class Meta:
        ordering = ['id']
 
 
# CONTROL MODEL
 
class Control(MPTTModel):
    """
    Controls that can be part of Evaluations
    """
    other_id = models.IntegerField(default=-1, unique=True)
    parent = TreeForeignKey('self', on_delete=models.CASCADE, null=True, blank=True, related_name='children')
    name = models.CharField(max_length=150, unique=True)
    descr = models.TextField(max_length=1000, default="none")  # Description of a node in the taxonomy
    TYPES = (
        ('cat', 'category'),
        ('con', 'control')
    )
    # Possible node type of the taxonomy (category node or control node)
    node_type = models.CharField(max_length=3, choices=TYPES, default='cat')
    target_type = models.ForeignKey(TargetType, blank=True, null=True, on_delete=models.CASCADE)  # It's null for the root node and category nodes
 
    def __str__(self):
        return str(self.name)
 
    class MPTTMeta:
        level_attr = 'level'
        order_insertion_by = ['name']
 
    class Meta:
        ordering = ['tree_id', 'lft']
 
 
# EVALUATION MODEL
 
class Evaluation(MPTTModel):
    """
    Evaluation is composed by one or more Controls, and can be used by Users
    """
    other_id = models.IntegerField(default=-1, unique=True)
    parent = TreeForeignKey('self', on_delete=models.CASCADE, null=True, blank=True, related_name='children')
    name = models.CharField(max_length=150, unique=True)
    descr = models.TextField(max_length=1000, default="none")  # Description of a node in the taxonomy
    TYPES = (
        ('cat', 'category'),
        ('eva', 'evaluation')
    )
    # Possible node types of the taxonomy (category node or evaluation node)
    node_type = models.CharField(max_length=3, choices=TYPES, default='cat')
    controls = models.ManyToManyField(Control)  # Evaluation can be composed of one or more controls
 
    def __str__(self):
        return str(self.name)
 
    class MPTTMeta:
        level_attr = 'level'
        order_insertion_by = ['name']
 
    class Meta:
        ordering = ['tree_id', 'lft']
 
 
# USER MODEL
 
class User(models.Model):
    """
    User registered to Moon Cloud with an email address, and can insert Target and launch Evaluations
    """
    other_id = models.IntegerField(default=-1, unique=True)
    email = models.EmailField(max_length=50, unique=True)
    evaluations = models.ManyToManyField(Evaluation, blank=True)  # Evaluations chosen by user
 
    def __str__(self):
        return str(self.email)
 
    class Meta:
        ordering = ['other_id', 'id']
 
 
# TARGET MODEL
 
class Target(models.Model):
    """
    Target (can be more than one) chosen by the user
    """
    user = models.ForeignKey(User, on_delete=models.CASCADE)  # User has chosen a target_type
    other_id = models.IntegerField(default=-1, unique=True)
    target_type = models.ForeignKey(TargetType, on_delete=models.CASCADE)  # TargetType Id
 
    def __str__(self):
        return str(self.user) + " " + str(self.other_id) + " " + str(self.target_type)
 
    class Meta:
        ordering = ['user']
\end{lstlisting}
%
A partire da questi Model vennero introdotte nel database le seguenti tabelle, le quali è possibile visionare nella Figura 
\ref{fig:str_db_project}.
\begin{description}
    \item[Control:] contiene l'insieme dei software, o soltanto i riferimenti, che vengono poi effettivamente eseguiti all'interno 
    di una Evaluation, i campi \texttt{other\_id} (identificativo dell'Evaluation che fa riferimento al database effettivo di Moon Cloud), 
    descr (una descrizione del funzionamento del controllo), \texttt{node\_type} (definisce se il nodo è un Evaluation o una Categoria) 
    definiscono le caratteristiche del controllo mentre lft, rght, \texttt{tree\_id}, \texttt{level} e \texttt{parent} sono introdotti 
    automaticamente dal package MPTT per poter rappresentare i dati in modo gerarchico, infine \texttt{target\_type\_id} rappresenta, quel 
    controllo a quale Target viene associato.
    \item[Evaluation:] contiene l'insieme di Evaluation che un utente può eseguire per un certo Target, e allo stesso modo 
    i campi contenuti nella tabella Control. La tabella intermedia \textit{evaluation\_controls}permette di memorizzare quali Controlli 
    sono associati a quali Evaluation.
    \item[User:] contiene gli utenti registrati alla piattaforma Moon Cloud, e sono anche loro, come con le tabelle precedenti, 
    identificati con un campo \texttt{other\_id}, e distinti da un email. La tabella intermedia \textit{user\_evaluations} permette di memorizzare 
    quali Evaluation un utente ha selezionato e usato.
    \item[Target:] contiene i  Target (o asset) un utente ha inserito e sui quali vuole effettuare dei processi 
    di monitoraggio e verifica, attraverso l'applicazione di politiche e Evaluation.
    \item[TargetType:] contiene i tipi di Target supportati da Moon Cloud. In generale sono supportati: \textit{URL}, 
    rappresentante l'URL dell'applicativo web, \textit{Host}, viene specificato l'indirizzo IP identificativo di un host e si distingue 
    tra sistema operativo \textit{Windows} o \textit{Linux} eseguito su esso, \textit{Aws}, rappresentante il servizio di cloud computing del 
    gruppo Amazon, e \textit{Azure}, definisce un servizio cloud fornito da Microsoft.
\end{description}
%
\begin{figure}[ht!]
    \centering
    \includegraphics[scale=0.6]{images/MoonCloudRecommendation_ER.png}
    \caption{Struttura del database.}
    \label{fig:str_db_project}
\end{figure}
%
\newpage
%
% VIEW A SCOPO DIDATTICO
%\hfill\break
\section*{Realizzazione delle View}
Successivamente per poter testare che la tassonomia creata per le Evalution e i Controlli fosse corretta e funzionante si è 
implementata un'interfaccia Web a scopo didattico.
Avviando il server, viene mostrata una home page la quale contiene una barra di navigazione da cui è possibile accedere,
attraverso il \textit{Admin}, alla admin page offerta da Django (successivamente personalizzata, per poter manipolare la base di dati, 
e aggiungere, eliminare item o utenti), mentre tramite il link \textit{Home} è possibile tornare a 
questa pagina. Il contenuto di questa pagina mostra una breve descrizione di un sistema di raccomandazione, e accedere tramite i due 
appositi pulsanti alle pagine specifiche per la navigazione della tassonomia delle Evaluation piuttosto che dei Controlli; tutto questo 
viene mostrato dalla Figura \ref{fig:MCRS_homepage} e il relativo Listing \ref{lst:view_homepage}.
%
\begin{figure}[ht!]
    \includegraphics[scale=0.3]{images/MCRS_homepage.png}
    \caption{Home page dell'applicativo web a scopo didattico.}
    \label{fig:MCRS_homepage}
\end{figure}
\lstset{style=python_code_style}
\begin{lstlisting}[language=Python, label=lst:view_homepage, caption={Parte principale del codice delle View della soluzione per gestire l'accesso 
    alla home page.}]
def index(request):
    """
    Index page where you can choose to navigate the evaluation taxonomy or the control taxonomy.
    :param request: HTTP request
    :return: HTTP response with the template to show to the user
    """
    return render(request, "recommendation_app/index.html")
\end{lstlisting}
%
Una volta scelta la tassonomia, di Controlli o delle Evaluation, su cui si vuole navigare, viene mostrata una pagina dove nella metà a sinistra 
troviamo la possibilità di selezionare un nodo della tassonomia e l'operazione che si vuole svolgere su quel nodo, inoltre, è possibile effettuare 
una richiesta di generazione di file in formato e linguaggio DOT e relativa immagine in formato .png, attraverso il pulsante \textit{Request schema}. 
Proseguendo più in basso, vengono mostrati tutti i nodi a cui è stato associato il tipo Categoria e il tipo Evaluation. Nella porzione destra della 
pagina web viene mostrata l'intera struttura della tassonomia e un link da quale è possibile accedere a una pagina di dettaglio che mostra una 
simile a quella presente sul database. Questa pagine è visibile dalla Figura \ref{fig:MCRS_taxindex} e dal Listing \ref{lst:view_tax_homepage}.\hfill\break
%
\begin{figure}[ht!]
    \includegraphics[scale=0.3]{images/MCRS_taxindex.png}
    \caption{Home page per la navigazione della tassonomia delle Evaluation.}
    \label{fig:MCRS_taxindex}
\end{figure}
%
\lstset{style=python_code_style}
\begin{lstlisting}[language=Python, label=lst:view_tax_homepage, caption={Home page dalla quale è possibile richiamare 
    diverse operazioni eseguibili sulla tassonomia.}]
def tax_index(request, taxonomy_used):
    """
    The home page shows all taxonomy and a form to make operations on it.
    :param request: HTTP request
    :param taxonomy_used: specify if it's used the Control taxonomy or the Evaluation taxonomy
    :return: HTTP response with the template to show to the user
    """
    # If this is a POST request we need to process the form data
    if request.method == 'POST':
        # Create a form instance and populate it with data from depending on the taxonomy_used
        if (taxonomy_used == "evaluation"):
            form = EvaluationOperationForm(request.POST)
        else:
            form = ControlEvaluationForm(request.POST)
        # Check whether it's valid:
        if form.is_valid():
            # Process the data in form.cleaned_data as required
            nodename_form = form.cleaned_data['nodeName']
            taxonomy_operation_form = form.cleaned_data['actionTax']
            # Redirect to a new URL (page that show a part of the taxonomy, depending on the action user has chosen):
            return redirect(
                reverse('rec:tax_index', args=[taxonomy_used]) + str(nodename_form) + '_' + taxonomy_operation_form)
    # If id's a GET method we'll create a blank form
    else:
        if (taxonomy_used == "evaluation"):
            form = EvaluationOperationForm()
        else:
            form = ControlEvaluationForm()
 
    # Depending on the taxonomy_used, I'm getting all the categories of Evaluations or Controls taxonomy and save it in a
    # list called "categories_list"
    if (taxonomy_used == "evaluation"):
        q_categories = Evaluation.objects.filter(node_type='cat')
    else:
        q_categories = Control.objects.filter(node_type='cat')
    categories_list = []
    for node in q_categories:
        categories_list.append(node.name)
 
    # Depending on the taxonomy_used, I'm getting all the categories of Evaluations or Controls node in the taxonomy
    # and save it in a list called "node_list"
    if (taxonomy_used == "evaluation"):
        q_nodes = Evaluation.objects.filter(node_type='eva')
    else:
        q_nodes = Control.objects.filter(node_type='con')
    node_list = []
    for node in q_nodes:
        node_list.append(node.name)
 
    # Depending on the taxonomy_used, I'm getting all the Evaluations or Controls taxonomy
    if (taxonomy_used == "evaluation"):
        tax = Evaluation.objects.all()
    else:
        tax = Control.objects.all()
 
    # Passing the complete taxonomy and data to fill the form so you can operate on the taxonomy
    args = {'tax': tax,
            'categories': categories_list,
            'nodes': node_list,
            'form': form,
            'request_path': taxonomy_used}
 
    return render(request, "recommendation_app/tax_index.html", args)
\end{lstlisting}
%
Le attività che è possibile svolgere sulla tassonomia, sia dei Controlli sia delle Evaluation, sono le seguenti, mentre le operazioni 
di manipolazione dei dati memorizzati dal database vengono svolte con l'ausilio della admin page messa a disposizione da Django.\hfill\break
\begin{itemize}
    \item Per ogni singolo nodo è possibile recuperare: i discendenti, i figli, la famiglia o i fratelli. Si ottiene un risultato come nella 
    Figura \ref{fig:MCRS_taxnodedetails} nel caso in cui si è scelti l'\textit{Evaluation Http robustness check} e si vuole ritornare tutta la 
    famiglia di quel nodo.
    %
    \begin{figure}[ht!]
        \includegraphics[scale=0.3]{images/MCRS_taxnodedetails.png}
        \caption{Risultato dell'operazione selezionata sul nodo in questione.}
        \label{fig:MCRS_taxnodedetails}
    \end{figure}
    %
    \lstset{style=python_code_style}
    \begin{lstlisting}[language=Python, label=lst:view_tax_nodedetails, caption={Codice utilizzato all'interno delle View per 
        implementare le operazioni per restituire i discendenti, i figli, la famiglia o i fratelli.}]
    # Methods to navigate the taxonomy
 
    def show_descendants(request, nodename, taxonomy_used):
        """
        Based on the MPTT's method 'get descendants' that return the descendants of a model instance, in tree order
        :param request: HTTP request
        :param nodename: name (it's unique for each node) of a node in the taxonomy
        :param taxonomy_used: specify if it's used the Control taxonomy or the Evaluation taxonomy
        :return: HTTP response with the template to show to the user
        """
        if (taxonomy_used == 'evaluation'):
            q_result = Evaluation.objects.get(name=nodename).get_descendants(include_self=False)
            # Get the count of descendants of the model instance
            q_result_num = Evaluation.objects.get(name=nodename).get_descendant_count()
        else:
            q_result = Control.objects.get(name=nodename).get_descendants(include_self=False)
            # Get the count of descendants of the model instance
            q_result_num = Control.objects.get(name=nodename).get_descendant_count()
 
        return render(request, "recommendation_app/tax_node_details.html",
                    {'tax_type': (str(taxonomy_used)).capitalize(),
                    'descendants': q_result,
                    'node_exe': nodename,
                    'method': 'descendants',
                    'num_descendants': q_result_num})
 
 
    def show_children(request, nodename, taxonomy_used):
        """
        Based on the MPTT's method 'get children' that return the immediate children of a model instance, in tree order
        :param request: HTTP request
        :param nodename: name (it's unique for each node) of a node in the taxonomy
        :param taxonomy_used: specify if it's used the Control taxonomy or the Evaluation taxonomy
        :return: HTTP response with the template to show to the user
        """
        if (taxonomy_used == 'evaluation'):
            q_result = Evaluation.objects.get(name=nodename).get_children()
        else:
            q_result = Control.objects.get(name=nodename).get_children()
 
        return render(request, "recommendation_app/tax_node_details.html",
                    {'tax_type': (str(taxonomy_used)).capitalize(),
                    'children': q_result,
                    'node_exe': nodename,
                    'method': 'children'})
 
 
    def show_family(request, nodename, taxonomy_used):
        """
        Based on the MPTT's method 'get family' that return the ancestors, the model instance itself and the descendants,
        in tree order
        :param request: HTTP request
        :param nodename: name (it's unique for each node) of a node in the taxonomy
        :param taxonomy_used: specify if it's used the Control taxonomy or the Evaluation taxonomy
        :return: HTTP response with the template to show to the user
        """
        if (taxonomy_used == 'evaluation'):
            q_result = Evaluation.objects.get(name=nodename).get_family()
        else:
            q_result = Control.objects.get(name=nodename).get_family()
 
        return render(request, "recommendation_app/tax_node_details.html",
                    {'tax_type': (str(taxonomy_used)).capitalize(),
                    'family': q_result,
                    'node_exe': nodename,
                    'method': 'family'})
 
 
    def show_siblings(request, nodename, taxonomy_used):
        """
        Based on the MPTT's method 'get siblings' that return siblings of the model instance (root nodes are considered
        to be siblings of other root nodes)
        :param request: HTTP request
        :param nodename: name (it's unique for each node) of a node in the taxonomy
        :param taxonomy_used: specify if it's used the Control taxonomy or the Evaluation taxonomy
        :return: HTTP response with the template to show to the user
        """
        if (taxonomy_used == 'evaluation'):
            q_result = Evaluation.objects.get(name=nodename).get_siblings()
        else:
            q_result = Control.objects.get(name=nodename).get_siblings()
 
        return render(request, "recommendation_app/tax_node_details.html",
                    {'tax_type': (str(taxonomy_used)).capitalize(),
                    'siblings': q_result,
                    'node_exe': nodename,
                    'method': 'siblings'})
    \end{lstlisting}
    \item Attraverso il pulsante \textit{request schema} nella home page della tassonomia si può effettuare una richiesta al software 
    Graphviz il quale attraverso un file scritto in linguaggio DOT genera lo schema della tassonomia sotto forma d'immagine in formato .png.\hfill\break
    Il DOT è un linguaggio descrittivo per grafi, tipicamente i file hanno estensione .gv o .dot, che in combinazione con il software Graphviz è possibile 
    effettuare il rendering della sintassi DOT in un immagine più significativa. Un esempio di tale linguaggio è mostrato nel Listing \ref{lst:DOT_eval} 
    viene mostrato un esempio di scrittura con linguaggio DOT.
    \lstset{style=python_code_style}
    \begin{lstlisting}[language=Python, label=lst:DOT_eval, caption={Codice parziale utilizzato per realizzare lo schema della tassonomia
        per le Evaluation.}]
    graph {
        1 [label="moon cloud"]
        12 [label="protocol"]
        1 -- 12
        11 [label="infrastructure"]
        1 -- 11
        4 [label="cloud"]
        1 -- 4
        16 [label="backup"]
        1 -- 16
        2 [label="database"]
        1 -- 2
        13 [label="web"]
        1 -- 13
        9 [label="human"]
        1 -- 9
        8 [label="custom"]
        1 -- 8
        3 [label="operating system"]
    }
    \end{lstlisting}
    %
    \lstset{style=python_code_style}
    \begin{lstlisting}[language=Python, label=lst:view_DOT_eval, caption={Codice utilizzato per la realizzazione del
        file .dot e relativa immagine .png.}]
    def dot_graph(request, taxonomy_used):
        """
        Create the .dot file (based on the Dot language) and the graph showing the taxonomy in .png format
        :param request: HTTP request
        :param taxonomy_used: specify if it's used the Control taxonomy or the Evaluation taxonomy
        :return: HTTP response with the template to show to the user
        """
        # Create a graph object
        taxonomy_dot_object = Graph(comment='Taxonomy', format='png')
        # Fill the graph with every node in the database (evaluations/controls node and categories nodes), 
        # and create a link with the parent node
        if (taxonomy_used == 'evaluation'):
            taxonomy_nodes = Evaluation.objects.all()
        else:
            taxonomy_nodes = Control.objects.all()
        i = 0
        for node in taxonomy_nodes.order_by('level'):
            # This If construct will prevent the adding of an empty node to the root node in the graph
            if (i == 0):
                # Insert the root node
                taxonomy_dot_object.node(str(node.id), label=str(node.name))
            else:
                # Insert the other nodes
                taxonomy_dot_object.node(str(node.id), label=str(node.name))
                taxonomy_dot_object.edge(str(node.parent_id), str(node.id))
            i += 1
        # Specify where I want to save the .png image and the .dot file
        taxonomy_dot_object.render('taxonomy_output/taxonomy.dot')
 
        # This function is used to zip a directory
        def make_zipdir(path, ziph):
            # Ziph is zipfile handle
            for root, dirs, files in os.walk(path):
                for file in files:
                    ziph.write(os.path.join(root, file))
 
        # Making the zip file
        zip_file = zipfile.ZipFile('taxonomy_output.zip', 'w', compression=zipfile.ZIP_DEFLATED)
        make_zipdir('taxonomy_output/', zip_file)
        zip_file.close()
 
        # Remove the directory which was zipped and all files inside
        shutil.rmtree("taxonomy_output/")
 
        return redirect(reverse('rec:index'))
    \end{lstlisting}
    \item Inoltre è possibile osservare in maniera più approfondita le informazioni rilevanti sulla tassonomia contenute nelle 
    tabelle del database; e nel caso, della tassonomia delle Evaluation, è possibile simulare il processo di Item Recommendation, 
    premendo il relativo tasto rispetto al nodo su cui si vuole determinare le altre Evaluation simili.
    %
    \begin{figure}[ht!]
        \includegraphics[scale=0.3]{images/MCRS_taxdetails.png}
        \caption{Dettagli della tassonomia sotto forma di tabella come nella base di dati.}
        \label{fig:MCRS_taxdetails}
    \end{figure}
    %
    \lstset{style=python_code_style}
    \begin{lstlisting}[language=Python, label=lst:view_taxdetails, caption={Codice utilizzato per la realizzazione della View che 
        implementa la pagina web del dettaglio della Tassonomia.}]
    def tax_details(request, taxonomy_used):
        """
        Show the taxonomy's details page showing an overview of the taxonomy
        :param request: HTTP request
        :param taxonomy_used: specify if it's used the Control taxonomy or the Evaluation taxonomy
        :return: HTTP response with the template to show to the user
        """
        if (taxonomy_used == 'evaluation'):
            tax_details_obj = Evaluation.objects.all()
        else:
            tax_details_obj = Control.objects.all()
 
        return render(request, "recommendation_app/tax_details.html",
                    {'tax_details': tax_details_obj,
                    'taxonomy_used': taxonomy_used})
    \end{lstlisting}
\end{itemize}
%
\newpage
%
% VIEW PER LE RACCOMANDAZIONI
\section*{View per i processi di raccomandazione}
% visto che il codice l'ho messo nel capitolo precedente, posso mettere qui degli esempi e mostrare le url per richiamare
% questi algoritmi, e nel caso dello User e Item algortihm, mostrare le view che li implementano come REST API 
% (non mostro quella dell'hybrid perché già presente al completo nel capitolo precedente)
In generale gli algoritmi di raccomandazione implementati in questo progetto, e descritti nel Capitolo \ref{chp:03-recommendationSystems}, 
vengono richiamati e utilizzati come API REST, un tipo di architettura basata sul protocollo HTTP. Un sistema REST, per funzionare, prevede 
una struttura degli URL ben definita (atta a identificare univocamente una risorsa o un insieme di risorse) l'utilizzo dei verbi HTTP specifici 
per il recupero d'informazioni (GET), per la modifica (POST, PUT, PATCH, DELETE) e per altri scopi (OPTIONS, ecc.).\hfill\break
In questo caso a ogni URL è associata una azione particolare la quale va a richiamare la View che permette di determinare un insieme di 
raccomandazioni sulla base di un parametro in ingresso, come è mostrato nel Listing \ref{lst:URL_rec}.\hfill\break
\lstset{style=python_code_style}
\begin{lstlisting}[language=Python, label=lst:URL_rec, caption={Porzione parziale del codice contenuto nell'URL mapper per implementare 
    i sistemi di raccomandazione.}]
# RECOMMENDATION VIEWS
 
# Item recommendation process
# Depending on the <str:item_other_id> value I decide on which evaluation of the taxonomy I want to recommend on
path('recommendation/item/<str:item_other_id>/', recommendation_views.item_recommendation,
        name='item_recommendation'),
 
# User recommendation process
# Depending on the <str:user_other_id> value I decide on which user I want to recommend on
path('recommendation/user/<str:user_other_id>/', recommendation_views.user_recommendation,
        name='user_recommendation'),
 
# Hybrid recommendation process (User and Item recommendation process)
# Depending on the <str:User_other_id> value I decide on which user I want to recommend on with an hybrid approach
path('recommendation/hybrid/<str:user_other_id>/', recommendation_views.hybrid_recommendation,
        name='hybrid_ui_recommendation'),
 
# Target recommendation process
# Depending on the <str:target_type_id> value I decide on which user I want to recommend on
path('recommendation/target/<str:target_type_id>/', recommendation_views.target_recommendation,
        name='target_recommendation'),
\end{lstlisting}
Sulla base dell'URL che viene richiamato e sul parametro (come \texttt{user\_other\_id}, \texttt{item\_other\_id}, ecc.) verranno fornite le 
relative raccomandazioni. Nel caso di questo progetto, le View che possono essere richiamate permettono soltanto chiamate HTTP con metodo GET 
e restituiscono una risposta HTTP in formato JSON, queste risposte contengono una lista di dizionari e ciascuno contiene l'\texttt{other\_id} della Evaluation 
che si vuole raccomandare. Di seguito nel Listing \ref{lst:view_rec} vengono mostrati le porzioni di codice rappresentanti 
le View per gli algoritmi di raccomandazione per Item (in questo caso Evaluaiton), Utenti e Target. Inoltre nel Listing \ref{lst:view_rec_ex} è 
possibile osservare alcuni esempi e risposte di richieste HTTP a questi algortmi.\hfill\break
\lstset{style=python_code_style}
\begin{lstlisting}[language=Python, label=lst:view_rec, caption={Porzione parziale del codice contenuto nelle View per implementare
    i sistemi di raccomandazione.}]
# Item recommendation API REST
@api_view(['GET'])
def item_recommendation(request, item_other_id):
    """
    Returns recommendations for the using an item-based recommendation algorithm for the Evaluation `item_other_id`
 
    :param request: http GET request
    :param item_other_id: value representing the other_id of an Item (Evaluation)
    :return: json response with the evaluations to recommend
    """
    # Trying to retrieve the actual node with item_other_id
    item = Evaluation.objects.get(other_id=item_other_id)
 
    similar_item_evaluations = item_recommendation_alg(item_other_id)
    # Cleaning the data, deleting all the keys except 'other_id'
    similar_item_evaluations_serilized = EvaluationSerializerRecommendation(similar_item_evaluations, many=True).data
 
    return JsonResponse(similar_item_evaluations_serilized, safe=False)
 
# User recommendation API REST
@api_view(['GET'])
def user_recommendation(request, user_other_id):
    """
    Returns recommendations for the using an user-based recommendation algorithm for the user `user_other_id`
 
    :param request: http GET request
    :param user_other_id: value representing the other_id of a User
    :return: json response with the evaluations to recommend
    """
    # Trying to retrieve the actual User with user_other_id
    user = User.objects.get(other_id=user_other_id)
 
    target_user_evaluations, similar_user_evaluations = user_recommendation_alg(user_other_id)
    # Cleaning the data, deleting all the keys except 'other_id'
    similar_user_evaluations_serilized = EvaluationSerializerRecommendation(similar_user_evaluations, many=True).data
 
    return JsonResponse(similar_user_evaluations_serilized, safe=False)
 
# Target Recommendation API REST
@api_view(['GET'])
def target_recommendation(request, target_type_id):
    """
    For a target (Target_Type like 'host', 'aws', 'url', etc.) chose by a user this algorithm search the possible
    evaluations that can be recommended for that user.
 
    :param request: http GET request
    :param target_type_id: identifier of a particular target_type
    :return: json response with the evaluations to recommend
    """
    # Trying to retrieve the actual Target Type, with id equal to target_type_id, chosed by a user
    target = TargetType.objects.get(id=target_type_id)
 
    target_evaluations = target_recommendation_alg(target_type_id)
 
    return JsonResponse(target_evaluations, safe=False)
\end{lstlisting}
\lstset{style=python_code_style}
\begin{lstlisting}[language=Python, label=lst:view_rec_ex, caption={Esempi di chiamate e risposte HTTP per i diversi algoritmi di raccomandazione.}]
Esempio di chiamata per lo Item Recommendation Algorithm
URL: http://127.0.0.1:8000/recommendation/item/34/ 
HTTP method: GET
 
Esempio di risposta per lo Item Recommendation Algorithm
HTTP Status: 200 OK
HTTP Response Body:
[
    {"other_id": 35},
    {"other_id": 36}
]
 
Esempio di chiamata per lo User Recommendation Algorithm
URL: http://127.0.0.1:8000/recommendation/user/10/ 
HTTP method: GET
 
Esempio di risposta per lo User Recommendation Algorithm
HTTP Status: 200 OK
HTTP Response Body:
[
    {"other_id": 36}
]
 
Esempio di chiamata per lo Hybrid Recommendation Algorithm
URL: http://127.0.0.1:8000/recommendation/hybrid/10/ 
HTTP method: GET
 
Esempio di risposta per lo Hybrid Recommendation Algorithm
HTTP Status: 200 OK
HTTP Response Body:
[
    {"other_id": 23},
    {"other_id": 24},
    {"other_id": 25},
    {"other_id": 26},
    {"other_id": 28},
    {"other_id": 29},
    {"other_id": 30},
    {"other_id": 31},
    {"other_id": 32},
    {"other_id": 35},
    {"other_id": 36},
    {"other_id": 43},
    {"other_id": 45},
    {"other_id": 46},
    {"other_id": 47},
    {"other_id": 48}
]
 
Esempio di chiamata per lo Target Recommendation Algorithm
URL: http://127.0.0.1:8000/recommendation/target/1/ 
HTTP method: GET
 
Esempio di risposta per lo Target Recommendation Algorithm
HTTP Status: 200 OK
HTTP Response Body:
[
    {"other_id": 25},
    {"other_id": 29},
    {"other_id": 30},
    {"other_id": 24},
    {"other_id": 28},
    {"other_id": 31},
    {"other_id": 23},
    {"other_id": 26}
]
\end{lstlisting}
%
\newpage
%
% ASPETTI DI DJANGO CHE HO PERSONALIZZATO, COME LE ADMIN PAGE
\section*{Personalizzazione Admin Page}
Per poter agilmente manipolare la base di dati, senza dover scrivere manualmente le query in puro Sql, Django mette a disposizione 
la cosiddetta Admin Page mostrata in Figura \ref{fig:MCRS_adminpage}, che è stata personalizzata per mostrare le tabelle su cui è possibile 
effetuare modifiche, e per ognuna vengono visualizzate le colonne più rilevanti, come mostrato dalla 
Figura \ref{fig:MCRS_adminpage_evaluationEX} nel caso della tabella Evaluation, e dalla quale è possibile effettuare ricerche, attraverso 
l'apposito campo, eliminare direttamente i record contenuti nel database e aggiungerne dei nuovi, come mostrato in 
Figura \ref{fig:MCRS_adminpage_evaluationEX_add}.\hfill\break
In particolare in questo ultimo caso, sulla base di come è stato costruito il Model della relativa tabella, si potranno avere dei campi 
la cui compilazione è opzionale e altri in cui è obbligatorio compilare per la creazione del record.
%
\begin{figure}[ht!]
    \includegraphics[scale=0.3]{images/MCRS_adminpage.png}
    \caption{Admin page creata automaticamente da Django.}
    \label{fig:MCRS_adminpage}
\end{figure}
%
\begin{figure}[ht!]
    \includegraphics[scale=0.3]{images/MCRS_adminpage_evaluationEX.png}
    \caption{Esempio di Admin page per la tabella delle Evaluation.}
    \label{fig:MCRS_adminpage_evaluationEX}
\end{figure}
%
\begin{figure}[ht!]
    \includegraphics[scale=0.55]{images/MCRS_adminpage_evaluationEX_add.png}
    \caption{Esempio di Admin page per il caso in cui si vuole aggiungere una nuova Evaluation.}
    \label{fig:MCRS_adminpage_evaluationEX_add}
\end{figure}
%
\newpage
% VIEW PER MANTERE LA CONSISTENZA COL MIO DATABASE
\section*{Consistenza tra i database}
% parlare di rest_framework, dei serializer, e url
Per fare in modo che il database creato in questo progetto e l'attuale database utilizzato dalla piattaforma di Moon Cloud 
fossero consistenti, quindi le informazioni presenti nell'uno siano identiche a quelle nell'altro, si è 
pensato d'implementare un sistema di API REST.\hfill\break
Nel caso vengano fatte modifiche, aggiunte o cancellazioni di dati relativi a Controlli, Evaluation, Target, User o Target Type 
è possibile richiamare tramite appositi URL, mostrati nel Listing \ref{lst:view_cos_rec_url}, queste funzioni ed eseguire la relativa 
operazione sul database implmentato nella soluzione, senza dover accedere e manipolare direttamente la base di dati.
\lstset{style=python_code_style}
\begin{lstlisting}[language=Python, label=lst:view_cos_rec_url, caption={Porzione di codice dell'URL Mapper contenti le URL usate per 
    mantenere la consistenza dei dati.}]
# API REST EXTERNAL to maintain database consistency
# Control views
url(r'^control/$', control_views.ControlListCreate.as_view(),
    name="control_list_create_view"),
url(r'^control/(?P<other_id>[0-9]+)/$', control_views.ControlRetrieveDeleteUpdate.as_view(),
    name="control_retrieve_delete_update_view"),
 
# Evaluation views
url(r'^evaluation/$', evaluation_views.EvaluationListCreate.as_view(),
    name="evaluation_list_create_view"),
url(r'^evaluation/(?P<other_id>[0-9]+)/$', evaluation_views.EvaluationRetrieveDeleteUpdate.as_view(),
    name="evaluation_retrieve_delete_update_view"),
 
# Target views
url(r'^target/$', target_views.TargetListCreate.as_view(),
    name="target_list_create_view"),
url(r'^target/(?P<other_id>[0-9]+)/$', target_views.TargetRetrieveDelete.as_view(),
    name="target_retrieve_delete_view"),
 
# User views
url(r'^user/$', user_views.UserListCreate.as_view(),
    name="user_list_create_view"),
url(r'^user/(?P<other_id>[0-9]+)/$', user_views.UserRetrieveDeleteUpdate.as_view(),
    name="user_retrieve_delete_update_view"),
url(r'^user_evaluations/$', user_views.UserEvaluationUpdate.as_view(),
    name="user_evaluations_update_view"),
\end{lstlisting}
%
% IMPLEMENTAZIONE CON DOCKER
\section*{Implementazione in Docker}
Una volta preparata la base e gli algoritmi di raccomandazione, con tutto ciò a essi collegati (View, URL, API REST); l'ultima fase nel ciclo di sviluppo è il 
deployment e la sua preparazione. In questo caso l'app viene deployata, come il resto dei componenti di Moon Cloud, in una serie di Container Docker. Per 
tale ragione, è stato predisposto il file \textit{docker-compose.yml}, il quale viene usato dalla Docker Machine per poter creare un Immagine contenente 
il necessario per l'esecuzione all'interno del Container, oltre alle caratteristiche del Container stesso e delle informazioni usate per poter 
comunicare con l'esterno.
%Codice del dockerfile + SPIEGAZIONE
\lstset{style=python_code_style}
\begin{lstlisting}[language=Xml, label=lst:docker_compose, caption={Contenuto del file docker-compose.yml per il deployment del sistema di raccomandazione 
    all'interno di un Container.}]
    version: '3'

    services:
      db:
        networks:
          - mooncloud
        image: postgres:12.1
        restart: always
        volumes:
          - "./init_db/initdb.sh:/docker-entrypoint-initdb.d/1-initdb.sh"
        environment:
          POSTGRES_USER: postgres
          POSTGRES_PASSWORD: postgres
    
      api:
        networks:
          - mooncloud
        build:
            dockerfile: Dockerfile
            context: .
        command: ./entrypoint.sh
        ports:
          - "8000:8000"
        depends_on:
          - db
        environment:
          DATABASE_HOST: db
          DATABASE_PORT: 5432
          DATABASE_USER: postgres
          DATABASE_PASSWORD: postgres
          DATABASE_NAME: mooncloud_rec
    
    networks:
      mooncloud:
        external: true
\end{lstlisting}
%
Questo file permette di configurare i servizi della propria applicazione; i quali, attraverso un singolo comando, possono essere creati e inizializzati a 
partire da questo singolo file.
La prima riga specifica sempre la versione (\textit{version}) del formato del file di Compose, successivamente si definiscono i \textit{services} e 
le \textit{networks}. In questo progetto si è fatto uso di due servizi (\textit{db} e \textit{api}), i quali speficiano le 
proprietà dei rispettivi Container uno per implementare il database (\textit{db}) e il secondo per le API REST (\textit{api}); nel primo caso per il 
processo di \textit{build} si fa uso di un'Immagine che è già stata pubblicata in un Docker Registry, mentre nel secondo caso essa viene costruita 
a partire dal codice sorgente scritto nel suo \textit{Dockerfile}. Per fare in modo che servizi siano accessibili dall'esterno il numero di porta 
(\textit{ports}) deve essere indicato, nel caso del sevizio \textit{api} ciò è specificato all'interno del \textit{Dockerfile}.\hfill\break
Le \textit{networks} definiscono le regole di comunicazione tra i Container, e tra i Container e gli host; delle reti in comune 
permettono a diversi servizi di essere visibili l'uno con l'altro.
%
\vspace{1.5 cm}
\hfill\break
Ogni qualvolta è stata portata a termine una fase dell'implementazione, proposta in questa tesi, sono state predisposte 
dei processi di verifica del funzionamento del codice scritto fino a quel momento. Questo genere di test, vennero fatti 
per essere certi che le varie fasi fossero in grado di funzionare insieme.\hfill\break
I primi test sono fatti sul database, accertandosi del suo funzionamento e di quello del package MPTT usato 
per costruire la tassonomia delle Evaluation e quella dei Controlli. Successivamente vennero fatte delle verifiche sulla 
corretta interazione, con la base di dati, attraverso le View a scopo didattico e che le modifiche fatte alle admin page 
standard di Django dessero i risultati attesi. Una volta implementati gli algoritmi di raccomandazione, e relative View, 
e il sistema di API REST per il mantenimento della consistenza tra il database, creato in questa soluzione, e quello 
attualmente in uso su Moon Cloud, venne creato un sistema di test automatizzato, grazie alle funzioni built-in di Django, 
in cui ogni test richiama la funzione da testare, salva il suo risultato, e lo confronta con quello atteso, 
oltre a verificare che il codice della risposta HTTP corrisponda a quello atteso.\hfill\break
Per semplificare lo sviluppo e la verifica del corretto funzionamento si è voluto creare dei test aggiuntivi tramite 
l'ausilio di Postman; una piattaforma nata per lo sviluppo e il test di API, inoltre semplifica molti step per il loro 
sviluppo così da poter creare API migliori e più velocemente. Inoltre permette di generare una documentazione 
machine-readable così da rendere le API più facili da usare.

%Durante la scrittura delle View e delle altre parti del codice, sono stati definiti dei test per poter garantire il funzionamento di queste 
%funzioni, anche nel caso di eventuali modifiche successive; questo grazie ai metodi già forniti da Django e all'ausilio di Postman.

% 5- Conclusioni
\chapter{Conclusioni}
\label{chp:05-conclusion}


%Presentazione dei risultati e conclusioni: la presentazione dei risultati dovrebbe consistere in una 
%descrizione tecnica dei risultati raggiunti,unitamente ad un commento critico e ad un’analisi della rispondenza agli 
%obiettivi iniziali (si consiglia pertanto di motivare 
%la rilevanza dei risultati e l’eventuale scostamento dagli obiettivi iniziali). La sezione relativa ai risultati dovrebbe 
%infine contenere una sintesi critica e un giudizio sull’esperienza effettuata, che renda conto di aspetti positivi e negativi
%per il tirocinante e per l’ente ospitante, del valore formativo, professionale e umano, così via.


%       CHIUSURA DELLA TESI     %
\backmatter

% Bibliografia
\printbibliography[heading=bibintoc, title={Bibliografia}]

\end{document}
