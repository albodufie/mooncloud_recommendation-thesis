\documentclass[12pt,fleqn,twoside,a4paper]{book}

\usepackage[T1]{fontenc} % codifica dei font
\usepackage[utf8]{inputenc} % lettere accentate da tastiera
\usepackage[italian]{babel} % lingua del documento
%\usepackage{epsfig}
\usepackage{graphicx}
\graphicspath{ {images/} }
%\usepackage{hyperref} % collegamenti ipertestuali e carica anche il paccheto 'url' (caricato per ultimo)

\usepackage{csquotes}
\usepackage[backend=biber, style=numeric, sorting=nty]{biblatex}
\addbibresource{bibliography.bib}

\usepackage{imakeidx}
\makeindex

\title{mooncloud recommendation system thesis}
\author{Andrea Michele Albonico}


\begin{document}

\frontmatter

\begin{titlepage}
    \begin{figure}
    	\centering
    	\includegraphics[height=5.0 cm]{logo.jpg}
    	\vspace{0.5 cm}
    \end{figure}
    \begin{center}
        {\Large Corso di Laurea in Informatica Musicale}
    \end{center}
    
    \begin{center}
        \vspace{3 cm}
        {\Large \textsc{Mooncloud recommendation system} }
    \end{center}
    \par
    \vspace{3 cm}
    \begin{flushleft}
        Relatore:\\ Claudio Agostino Ardagna\\
        \noindent Correlatore:\\ Nome COGNOME
    \end{flushleft}
    \vspace{1 cm}
    \begin{flushright}
        Tesi di Laurea di:\\ Andrea Michele Albonico\\ Matricola: 886667
    \end{flushright}
    \vfill
    \begin{center}
    	{\Large Anno Accademico 2019/2020}
    \end{center}
\end{titlepage}

% RINGRAZIAMENTI 
\chapter*{Ringraziamenti}



\begin{flushright}
    \textit{Andrea Michele Albonico}
\end{flushright}

% motivazione con breve descrizione del progetto
\chapter{Prefazione}
I sistemi di raccomandazione (\textit{Recommendation System}) hanno avuto un forte sviluppo negli ultimi decenni e 
nascono proprio con lo scopo di identificare quegli oggetti (detti generalmente \textit{item}) all'interno di vasto 
mondo di informazioni che possono essere di nostro interesse e tanto maggiore è il grado di conoscenza dell'individuo 
e tanto più vengono ritenuti affidabili.
\newline
Il motivo di questo successo risiede nella riuscita integrazione di tali sistemi in applicazioni commerciali, 
soprattutto nel mondo dell’E-commerce e nel fatto che sono in grado di aiutare un utente a prendere una decisione 
che sia la scelta di un film per l'uscita con gli amici il sabato sera, di una playlist da ascoltare durante 
un viaggio in auto o in un momento di lettura, e via discorrendo. 
\newline
MoonCloud è una piattaforma erogata come servizio che fornisce un meccanismo di \textit{Security Governance} centralizzato. 
Garantisce il controllo della sicurezza informatica in modo semplice e intuitivo, attraverso attività di test e monitoraggio 
periodiche e programmate (\textit{Security Assurance}). L'obbiettivo di questa tesi è stato quello di aggiungere, al già 
presente sistema per la scelta dei controlli all'interno delle attività di test, un sistema di raccomandazioni che possa 
consigliare all'utente delle possibili \textit{evaluation} rispetto ai dati relativi al target indicato; in questo modo 
anche l'utente meno esperto può usufruire dei servizi offerti da MoonCloud in modo semplice e intuitivo.  


La tesi \`e organizzata come segue:
\begin{description}
    \item[Capitolo 1 -- Introduzione a MoonCloud] descrizione e funzionamento della piattaforma MoonCloud. 
    \item[Capitolo 2 -- Descrizione delle attività preliminari] studi e analisi di soluzioni esistenti, studi delle tecnologie utilizzate
    nel seguito del lavoro.
    \item[Capitolo 3 -- ] Descrizione delle attività svolte per conseguire gli obiettivi: Descrivere le attività svolte, 
    riportando attivita, tempi, strumenti utilizzati, risultati conseguiti, problemi affrontati e modalita di risoluzione. 
    Potranno essere qui descritte le attività anche dal punto di vista strettamente tecnico, approfondendo le scelte effettuate, 
    le motivazioni, le alternative prese in considerazione, l’uso o il possibile uso dei risultati del lavoro.
    \item[Capitolo 4 -- ] Presentazione dei risultati e conclusioni] La presentazione dei risultati dovrebbe consistere in una 
    descrizione tecnica dei risultati raggiunti,unitamente ad un commento critico e ad un’analisi della rispondenza agli 
    obiettivi iniziali (si consiglia pertanto di motivare 
    la rilevanza dei risultati e l’eventuale scostamento dagli obiettivi iniziali). La sezione relativa ai risultati dovrebbe 
    infine contenere una sintesi critica e un giudizio sull’esperienza effettuata, che renda conto di aspetti positivi e negativi
    per il tirocinante e per l’ente ospitante, del valore formativo, professionale e umano, così via.
\end{description}






% INDICE DELLA TESI
\tableofcontents
\listoffigures


%       CONTENUTO DELLA TESI        %
\mainmatter

% 1- Stato dell'arte e introduzione con descrizione più approfondita
\chapter{Introduzione}
In questo capitolo verrà descritto in modo più approfondito 
il funzionamento della piattaforma Moon Cloud e unitamente al 
motivo dell'implementazione della soluzione proposta.

\section{Perchè Moon Cloud?}
La diffusione di sistemi ICT (\textit{Information and Communications Technology})
nella maggiorparte degli ambienti lavorativi e privati in termini di servizi offerti,
automazione di processi e incremento delle performance. L'uso di questa tecnologia 
ha assunto importanza a partire dagli anni novanta come effetto del boom di Internet.
Oggi le professionalità legate all'ICT crescono in numero e si evolvono per 
specificità, per operare in ambienti fortemente eterogenei ma sempre più interconnessi 
fra di loro come cloud computing, social newtwork, marketing digitale, IoT, realtà virtuale,
ecc.

Il prezzo che paghiamo per i benefici di queste tecnologie è dato dall'incremento di 
violazione di sicurezza, che oggigiorno preoccupa tutte le aziende, e di conseguenza anche i 
loro clienti, con l'incremento del rischio di fallimento per i servizi più importanit, 
violazioni della privacy e furto di dati.
\newline
Il mercato sta lentamente notando che non è l'inadequamento tecnologico dei sistemi di sicurezza
che incrementa il rischio di furti di dati o violazioni di sicurezza; piuttosto, la mal configurazione
e errata integrazione di questi sistemi nei processi di business sono la base per 
i furti e le violazioni. \cite{cloud-Platform-for-ICT-Security-Governance}

Per questo motivo anche se vengono usati i sistemi di sicurezza e di controllo migliori non è possibile 
garantire la sicurezza; ma è necessario implementare un processo continuo di diagnostica che verifica che 
controlli sono configurati in modo corretto e il loro comportamento è quello aspettato.

Security assesment diventa allora un aspetto importante specialmente negli ambienti cloud e IoT. Questo 
assesment deve essere fatto in modo continuo e olistico, per correlare le prove raccolte da sempre maggiori 
meccanismi di protezione
\newline
Moon Cloud è una soluzione PaaS (Platform as a Service) che fornice una piattaforma B2B innovativa per verifiche, 
diagnostiche e monitoraggio dell'adeguatezza dei sistemi ICT rispetto alle politiche di sicurezza, in modo continuo 
e su larga scala.
Moon Cloud supporta una semplice ed efficente \textit{ICT security governance}, dove le politiche di sicurezza possono
essere definite dalle compagnie stesse (a partire da un semplice controllo sulle vulnerabilità a linee guida di
sicurezza interna), da entità esterne, imposte da standard oppure da regolamentazioni nazionali/internazionali.
\newline
La sicurezza di un sistema o di un insieme di asset dipende solo parzialmente dalla forza dei
singoli meccanismi di protezione isolati l'uno dall'altro; infatti, dipende anche dall'abilità di questi meccanismi 
di lavorare continuamente in sinergia per provvedere a una protezione olistica.
In più, quando i sistemi cloud e i servizi IoT sono coinvolti, le dinamiche di questi servizi e la loro rapida 
evoluzione rende il controllo dei processi all'interno dell'azienda e le politiche di sicurezza 
più complesse e prone ad errori.
\newline
I requisiti ad alto livello fondamentali per poter garantire le security assurance sono:
\begin{description}
	\item[sistema olistico] è richiesta una visione globale e pulita dello stautus dei sistemi di sicurezza; 
	inoltre è cruciale distribuire lo sforzo degli specialisti in sicurezza per migliorare il processo e le 
	politiche messe in atto. Si parte da della valutazioni fatte manualmente a quella semi-automatiche che
	ispezionano i meccanismi di sicurezza. 
	\item[monitoraggio continuo ed efficiente] è necessario un controllo continuo che valuti l'efficenza dei 
	sistemi di sicurezza per ridurre l'impatto dell'errore umano, soprattutto dal punto di vista organizzativo.
	La mancata configurazione dovuta al cambiamento dell'ambiente, la coesistenza di componenenti in conflitto: 
	sono scenari che richiedono un monitoraggio e un aggiornamento continuo.
	\item[singolo punto managment] avere un solo punto in cui gestire tutti gli aspetti relativi alla sicurezza,
	permette di avere sotto controllo le politiche di sicurezza, Inoltre disporre di un inventario degli asset 
	da proteggere, così da poter conoscere quali protezioni applicare.
	\item[reazioni rapide a incidenti di sicurezza] spesso la reazione ad incidenti di sicurezza è ritardata 
	da due fattori: il tempo richiesto per rilevare l'incidente e il tempo per analizzare il motivo dell'accaduto.
\end{description}


Moon Cloud è basato su una tecnica di security assurance garantendo che tutti le attività aziendali si compiano
seguendo i requisiti prestabiliti da appropriate politiche e procedure.

Una security compliance evaluation è il processo di verifica che un target è sottoposto il cui risultato deve 
soddisfare i richiesti standard e politiche. Da questi processi di verifica, che devono a loro volta essere 
affidabili, si ottendo delle evidence (prove); queste ultime possono essere raccolte monitorando l'attività
del target oppure, come già menzionato, sottoponendo il target a scenari
critici o testing.
In particolare, una security compliance evaluation è un processo che verifica la compliance (uniformità)
di un certo target a una o più politiche; vengono eseguiti tutti i controlli e produce un valore booleano per
le politiche, in base al valore booleano associato ad ogni controllo.

% 2- Tecnologie
\chapter{Tecnologie}
In questo capitolo verranno descritte le attività preliminari per la realizzazione di questo progetto, le tecnologie utilizzate
unitamente alle motivazioni legate all'uso di questi sistemi rispetto ad altri.

\section{Strutture dati gerarchiche}
Le tabelle di un database relazione non sono gerarchiche (come nel XML), ma sono delle semplici liste piatte. I dati gerarchici sono 
constituiti da relazioni padre-figlio che non possono essere rappresentate in modo naturale nelle tabelle dei database relazionali.
In questo caso, i dati gerarchici sono una collezione di informazioni dove ogni item ha un solo padre e nessuno o più figli
(ad eccezione del nodo radice che non ha un nodo padre); questo genere di rappresentazione delle informazioni può essere trovato in 
diversi ambiti di applicazione di un database, incluse discussioni su forum e mailing list, grafici di organizzazione di un business, 
categorie per gestire contenuti e categorie di prodotti. 

\begin{figure}[H]
	\label{fig:Hde}
	\includegraphics[scale=0.45]{images/Hierarchical_Data_ex.PNG}
	\caption{Esempio di una gestione di dati in modo gerarchico}
\end{figure}

Ci sono differenti modelli per poter gestire dati in modo gerarchico, i più importanti che sono stati presi in considerazione sono i 
seguenti:

\newpage

\subsection{The adjacency list model}
Il primo approccio, e quello di più semplice implementazione, qui descritto è chiamato \textit{‘adjacency list model} o metodo ricorsivo;
è definito tale perchè per funzionare necessita solo di una funzione che itera per tutto l'albero.

In questo modello, ogni item (nodo dell'albero) nella tablla contiene un puntatore al suo item padre; invece il nodo radice avrà un puntatore a un valore
NULL per l'item padre.

\begin{figure}[ht!]
    \centering
	\includegraphics[scale=0.75]{images/Adjacency_list_model_table.PNG}
	\caption{Esempio di una tabella per gestire dati in modo gerarchico secondo l' adjacency list model }
\end{figure}
 
Il vantaggio di usare questo modello sta nella sua semplicità di costruzione sopratutto a livello di codice client-side, 
e di restituzione dei figli di un nodo. Mentre diventa problematico se si lavora in puro SQL e nella maggior parte dei linguaggi di 
programmazione, è lento e poco efficente, perchè è necessaria una query per ogni nodo dell'albero, e visto che ogni query impiega 
un certo periodo di tempo, questo rende la funzione molto lente quando si lavora con alberi di grandi dimensioni.
Inoltre molti linguaggi non sono ottimizzati per funzioni ricorsive. Per ogni nodo, la funzione inizia una nuova istanza di se stessa,
ogni istanza occupa una porzione di memoria e impiega un certo tempo per inizializzarsi, e più grande è l'albero e più questo 
processo sarà portato a termine in maggior tempo.

\subsection{The Nested set model}
Il secondo approccio che viene proposto è il \textit{Nested set model}, che permette di osservare la gerarchia in un modo diverso, non 
come nodi e linee, ma come container innestati.

\begin{figure}[ht!]
    \centering
	\includegraphics[scale=0.55]{images/Nested_Tree_Model_ex.PNG}
	\caption{Esempio di una gestione di dati in modo gerarchico secondo il Nested set model}
\end{figure}

La gerarchia dei dati viene rappresentata nella tabella attraverso l'uso degli attributi 'left' e 'right' per rappresentare l'annidamento
dei nodi (il nome delle colonne: left e right, hanno significati speciali in SQL; per questo motivo si identificano questi campi con i 
nomi 'lft' e 'rght'). 
Ogni nodo dell'albero viene visitato due volte, assegnando i valori in ordine di visita, e in entrambe le visite. Quindi vengono 
associati ad ogni nodo due numeri, memorizzato come due attributi. 
I valori di left e right sono determinati come segue: si inizia a numerare a partire dal lato più a sinistra di ogni nodo e si continua 
verso destra. Lavorando con un albero, si parte da sinistra e si continua verso destra, un livello alla volta, scendendo per ogni
nodo i suoi figli, assegnando i valori al campo left, prima di assegnare un valore al campo right, e successivamente si continua verso 
destra.
Questo approccio è chiamato Modified preorder tree traversal algorithm.

A prima vista questo approccio può sembrare più complicato da comprendere rispetto all'adjacency list model, ma quest'utlimo metodo è
molto più veloce quando si vuole recuperare i nodi, visto che basta una query, mentre più lento per operazioni di aggiornamento e 
cancellazione dei nodi; in quest ultimo il grado di complicatezza dell'operazione è determinato dal nodo che si vuole cancellare, a 
partire dal caso più semplice, il nodo foglia (nodo senza figli) fino al caso più complicato, quando si vuole cancellare il nodo radice.

\begin{figure}[ht!]
    \centering
	\includegraphics[scale=0.6]{images/Nested_Tree_Model_table.PNG}
	\caption{Esempio di una tabelle per la gestione di dati in modo gerarchico secondo il Nested set model}
\end{figure}

\newpage

\section{Sistemi di raccomandazione}
Un sistema di raccomandazione filtra i dati usando differenti algoritmi e raccomanda gli item più rilevanti agli utenti,
attraverso un procedimento a 3 fasi:

\begin{description}
	\item[raccolda di dati]: questa è il primo step e anche quello più importante per poter costruire un sistema di 
	raccomandazione che produca risultanti rilevanti e consistenti. I dati possono essere raccolti in due modi: esplicitamente,
	cioè attraverso i dati che vengono prodotti direttamente dagli utenti, ad esempio le valutazioni di un prodotto; mentre 
	attarverso l'approccio implicito, vengono raccolti dati che non sono prodotti in modo intenzionale dall'utente ma raccolti
	dai costanti flussi di dati come la cronologia di ricerca, i click effettuati, lo storico degli ordini, etc.
	\item[memorizzazione di dati]: la quantità di dati definisce quanto efficace un modello di raccomandazione possa di
	diventare. Ad esempio, in un sistema di raccomanzione per film, maggiori sono le valutazioni fornite dagli utenti, e 
	migliore sarà il sistema di raccomandazione per gli altri utenti. Il tipo di dati che si vuole raccogliere determina
	anche il supporto di memorizzazione più adatto.   
	\item[Filtraggio dei dati]: dopo la fase di raccolta e memorizzazione dei dati, essi vanno filtrati per poter estrarre
	le informazioni rilevanti per poter effettuare le raccomandazioni finali, e sono già disponibili diversi algoritmi che
	semplificano quest ultima fase. 
\end{description}

I sistemi di raccomandazione possono essere suddivisi nelle seguenti categorie:

\subsection{Content-based filtering}
Un Content-based filtering è un sistema di raccomandazione in cui vengono suggeriti item simili a un particolare item 
(oggetti o prodotti). 

Questo approccio sfrutta i metadati dell'item, che possono essere il genere, una descrizione, uno o più autori, etc. per 
fare queste raccomandazioni; l'idea base che sta dietro questi raccomandatori, è che se ad un utente piace o interessa
un particolare item allora gli piaceranno anche altri item simili.

Questo algoritmo suggerisce prodotti che piacevano all'utente nel passato ed è limitato a item dello stesso tipo. Un 
content-based recommender fa riferimento a quegli approcci, che provvedono raccomandazioni comparano la rappresentazione del
contenuto che descrive un item e la rappresentazione del contenuto dell'item interessato dall'utente. 

Questi metodi sono usati quando si sanno a priori delle informazioni sugli item che si vuole suggerire, ma non sugli utenti.
In questo sistema, delle keyword (parole chiave) sono utilizzate per caratterizzare gli item e un profilo dell'utente è 
costruito per indicare quali item gli piacciono. In altre parole, questi algoritmi cercano di raccomandare item che 
all'utente sono piaciuti o ha usato nel passato e sta esaminando nel presente. La costruzione del profilo dell'utente,
spesso temporaneo, non viene basata su un modulo di registrazione che l'utente stesso deve compilare, ma su informazioni
lasciate indirettamente dall'utente. Più precisamente, tra vari item candidati da raccomandare all'utente si passa per un 
processo di confronto con gli item piaciuti dall'utente e gli item migliori vengono suggeriti.



\subsection{Collaborative-based filtering}


\paragraph{Memory-based}


\subparagraph{User-based filtering}


\subparagraph{Item-based filtering}


\paragraph{Model-based}


\subsection{Hybrid recommendation system}


\subsection{Challenges and limitations}



% 3- Recommendation systems
\chapter{Sistemi di raccomandazione}

% 4- Descrizione del mio progetto
\chapter{Descrizione approfondita del progetto}

% 5- Conclusioni
\chapter{Conclusioni}


%       CHIUSURA DELLA TESI     %
\backmatter

% Bibliografia
\printbibliography[heading=bibintoc, title={Bibliografia}]

\end{document}
