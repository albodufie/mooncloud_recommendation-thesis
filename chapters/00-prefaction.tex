\chapter{Prefazione}\label{chp:00-prefaction}
I sistemi di raccomandazione (\textit{Recommendation System}) hanno avuto un forte sviluppo negli ultimi decenni e
nascono proprio con lo scopo d'identificare quegli oggetti (detti generalmente \textit{item}) all'interno di un vasto 
mondo d'informazioni che possono essere di nostro interesse e tanto maggiore è il grado di conoscenza dell'individuo 
e tanto più vengono ritenuti affidabili.\hfill\break
Il motivo di questo successo risiede nella riuscita integrazione di tali sistemi in applicazioni commerciali, 
soprattutto nel mondo dell’E-commerce e nel fatto che sono in grado di aiutare un utente a prendere una decisione, che sia la scelta di 
un film per l'uscita con gli amici il sabato sera, di una playlist da ascoltare durante un viaggio in auto o in un momento di lettura, 
e via discorrendo.\hfill\break
Moon Cloud è una piattaforma erogata come servizio che fornisce un meccanismo di \textit{Security Governance} centralizzato. 
Garantisce il controllo della sicurezza informatica in modo semplice e intuitivo, attraverso attività di test e monitoraggio 
periodiche e programmate (\textit{Security Assurance}). L'obbiettivo di questa tesi è stato quello di aggiungere, al già 
presente sistema per la scelta dei Controlli all'interno delle attività di test, un sistema di raccomandazione che possa 
consigliare all'utente delle possibili \textit{Evaluation} rispetto al target indicato; in questo modo anche l'utente meno esperto può 
usufruire dei servizi offerti da Moon Cloud in modo semplice e intuitivo.  
\vspace{0.5 cm}
\hfill\break
La tesi è organizzata come segue:
\begin{description}
    \item[Capitolo 1 -- Introduzione a Moon Cloud] in questo capitolo viene descritta la piattaforma Moon Cloud e il suo funzionamento
    in ambito di Security Assurance. 
    \item[Capitolo 2 -- Tecnologie utilizzate] in questo capitolo vengono presentati gli studi e le analisi di soluzioni esistenti, 
    studi delle tecnologie utilizzate per la realizzazione del progetto.
    \item[Capitolo 3 -- Collaborative filtering] in questo capitolo viene descritto in modo più approfondito gli studi compiuti sui
    Filtri Collaborativi che hanno portato alla realizzazione dei sistemi di raccomandazione proposti nella soluzione implementata 
    per la piattaforma Moon Cloud, inoltre verranno mostrate le relative porzioni di codice. 
    \item[Capitolo 4 -- Descrizione della soluzione] in questo capitolo viene descritta in maniera dettagliata la realizzazione 
    dell'applicativo, analizzando quali sono state le difficoltà maggiori, i risultati ottenuti e l'uso che se ne è fatto. 
    \item[Capitolo 5 -- Conclusioni] in questo capitolo vengono esposte le conclusioni e i possibili sviluppi futuri delle attività
    svolte e del sistema realizzato.
\end{description}
