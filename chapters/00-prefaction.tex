\chapter{Prefazione}
I sistemi di raccomandazione (\textit{Recommendation System}) hanno avuto un forte sviluppo negli ultimi decenni e nascono proprio con lo scopo di identificare quegli oggetti (detti generalmente \textit{item}) all'interno di vasto mondo di informazioni che possono essere di nostro interesse e tanto maggiore è il grado di conoscenza dell'individuo e tanto più vengono ritenuti affidabili.\\
Il motivo di questo successo risiede nella riuscita integrazione di tali sistemi in applicazioni commerciali, soprattutto nel mondo dell’E-commerce e nel fatto che sono in grado di aiutare un utente a prendere una decisione che sia la scelta di un film per l'uscita con gli amici il sabato sera, di una playlist da ascoltare durante un viaggio in auto o in un momento di lettura, e via discorrendo. 


MoonCloud è una piattaforma erogata come servizio che fornisce un meccanismo di \textit{Security Governance} centralizzato. Garantisce il controllo della sicurezza informatica in modo semplice e intuitivo, attraverso attività di test e monitoraggio periodiche e programmate (\textit{Security Assurance}). L'obbiettivo di questa tesi è stato quello di aggiungere, al già presente sistema per la scelta dei controlli all'interno delle attività di test, un sistema di raccomandazioni che possa consigliare all'utente delle possibili \textit{evaluation} rispetto ai dati relativi al target indicato; in questo modo anche l'utente meno esperto può usufruire dei servizi offerti da MoonCloud in modo semplice e intuitivo.  


La tesi \`e organizzata come segue:
\begin{description}
    \item[Capitolo 1 -- ]
\end{description}




