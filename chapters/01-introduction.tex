\chapter{Scenario e motivazioni}\label{chp:01-overview}
In questo capitolo viene descritto in modo più approfondito il funzionamento della piattaforma Moon Cloud unitamente al 
motivo dell'implementazione della soluzione proposta.
%
\section{Introduzione}
La diffusione di sistemi \textit{Information and Communications Technology} (definito anche con l'acronimo ICT) ha avuto luogo nella 
maggior parte degli ambienti lavorativi e privati in termini di servizi offerti, automazione di processi e incremento delle performance. 
L'uso di questa tecnologia ha assunto importanza a partire dagli anni novanta come effetto del boom d'Internet e al giorno d'oggi le 
professionalità legate almondo dell'ICT crescono in numero e si evolvono per specificità, per operare in ambienti fortemente eterogenei 
ma sempre più interconnessi fra di loro come il Cloud Computing, i Social Newtwork, il Marketing Digitale, i Sistemi IoT, 
la Realtà Virtuale, ecc.
\vspace{0.5 cm}
\hfill\break
In particolare, il Cloud Computing ha portato un rivoluzionario paradigma nella creazione di un nuovo business, virtuale e accessibile, 
in qualunque momento e luogo; esso sfrutta le tecnologie messe a disposizione dai sistemi ICT come le operazioni di virtualized computing, 
internet e distributed computing, provvedendo un sistema integrato molto potente. Google, Microsoft, Amazon sono un esempio di 
aziende che forniscono servizi di Cloud Computing in business ICT. Si può definire il Cloud Computing come l'abilità di accedere a 
risorse (come database o applicazioni) in poco tempo e in tutto il mondo attraverso una rete.\hfill\break
Gli immensi benefici del Cloud in termini di flessibilità, consumo delle risorse e gestione semplificata, lo rendono la prima scelta per 
utenti e industrie per il deploy dei loro sistemi IT. Tuttavia il Cloud Computing solleva diverse problematiche legate alla mancanza di 
fiducia e trasparenza dove i clienti necessitano di avere delle garanzie sui servizi Cloud ai quali si affidano; spesso i fornitori di 
questi servizi non fornisco ai clienti le specifiche riguardanti le misure di sicurezza messe in atto.\hfill\break
Negli ultimi anni, sono state sviluppate tecniche e modi per rendere sicuri questi sistemi e proteggere i dati degli utenti, portando 
alla diffusione di approcci eterogenei che incrementarono la confusione negli utenti.
Tecniche tradizionali di verifica della sicurezza basati su metodi di analisi statistica non sono più sufficienti e devono essere integrati 
con processi di raccolta di prove (in inglese \textit{evidence}) da sistemi Cloud in produzione e funzionanti.
In generale la \textit{Cloud Security} definisce i modi, come criptazione e controllo degli accessi, per proteggere attivamente gli asset 
da minacce interne ed esterne, e fornire un ambiente in cui i clienti possano affidarsi e interagire in totale sicurezza.\hfill\break
Tutto questo non basta a rendere il Cloud fidato e trasparente, per questo sono state introdotte tecniche di
\textit{Security Assurance}, delle garanzie che permettono di ottenere la fiducia necessaria nelle infrastrutture e/o nelle 
applicazioni di dimostrare il rispetto di certe proprietà di sicurezza, e che operino normalmente anche se subiscono attacchi; grazie 
alla raccolta e allo studio di evidence è possibile che venga accertata la validità e l'efficienza delle proprietà di sicurezza messe in 
atto.\hfill\break
Il prezzo che si paga per i benefici di questa tecnologia è dato dall'incremento di violazioni di sicurezza, che oggigiorno 
preoccupa tutte le aziende e di conseguenza anche i loro clienti, con l'incremento del rischio di fallimento per i servizi più importanti 
dovuti a violazioni della privacy e al furto di dati.
Il mercato sta lentamente notando che non è l'inadeguatezza tecnologica dei sistemi di sicurezza che incrementa il rischio delle 
violazioni; piuttosto, la mal configurazione e l'errata integrazione di questi sistemi nei processi di business 
\cite{cloud-Platform-for-ICT-Security-Governance}.\hfill\break
%
\section{Security Assurance}
L'utilizzo di sistemi di sicurezza e di controllo migliori non garantisce in modo assoluto la sicurezza dell'infrastruttura; 
per garantire ciò è necessario implementare un processo continuo di diagnostica e verifica della corretta configurazione dei Controlli, 
supervisionando il loro comportamento, accertandosi che sia quello aspettato.
\vspace{0.5 cm}
\hfill\break
La \textit{Security Assessment} diventa allora un aspetto importante specialmente negli ambienti Cloud e IoT. Questo processo, costituito
da un insieme di attività mirate alla valutazione del rischio in sistemi IT, deve essere portato avanti in modo continuo e olistico, per 
correlare le evidence raccolte da sempre maggiori meccanismi di protezione \cite{mooncloud-semi-automatic-and-trustworthy}.\hfill\break
In più, quando i sistemi Cloud e i servizi IoT sono coinvolti, le dinamiche di questi servizi e la loro rapida evoluzione rende il 
controllo dei processi all'interno dell'azienda e le politiche di sicurezza più complesse e prone ad errori.
\vspace{0.5 cm}
\hfill\break
I requisiti ad alto livello fondamentali per poter garantire la Security Assurance sono i seguenti.
\begin{description}
    \item[Sistema olistico] è richiesta una visione globale e pulita dello status dei sistemi di sicurezza; inoltre, è cruciale 
    distribuire lo sforzo degli specialisti in sicurezza per migliorare il processo e le politiche messe in atto. Si parte da 
    delle valutazioni fatte manualmente a quella semi-automatiche che vengono usate per ispezionare i meccanismi di sicurezza. 
    \item[Monitoraggio continuo ed efficiente] è necessario un controllo continuo che valuti l'efficienza dei sistemi di sicurezza 
    per ridurre l'impatto dell'errore umano, soprattutto dal punto di vista organizzativo. La coesistenza di componenti in conflitto o
    la mancata configurazione dovuta al cambiamento dell'ambiente possono essere scenari che richiedono un monitoraggio e un 
    aggiornamento continuo.
    \item[Singolo punto di management] avere un solo punto d'accesso in cui poter gestire tutti gli aspetti relativi alla sicurezza, 
    permette di avere sotto controllo le politiche di sicurezza. Inoltre, disporre di un inventario degli asset da proteggere permette di
    poter conoscere quali meccanismi di protezioni applicare.
    \item[Reazioni rapide a incidenti di sicurezza] spesso la reazione a queste situazioni è ritardata da due fattori: il tempo 
    richiesto per rilevare l'incidente e il tempo per analizzare il motivo dell'accaduto; e avere un sistema che implementa un monitoraggio
    continuo permette di venire a conoscenza di questi problemi in breve tempo e agire di conseguenza.
    \label{list:security-assurance-fondamentals}
\end{description}
%
\section{Moon Cloud}
Moon Cloud è una soluzione PaaS (acronimo inglese di \textit{Platform as a Service}) che fornice una piattaforma B2B 
(\textit{Business To Business}) innovativa per verifiche, diagnostiche e monitoraggio dell'adeguatezza dei sistemi ICT, in modo continuo e su larga scala,
rispetto alle politiche di sicurezza. Essa supporta una semplice ed efficiente \textit{ICT Security Governance}, dove le politiche 
di sicurezza possono essere definite dalle compagnie stesse (a partire da un semplice controllo sulle vulnerabilità a linee guida di
sicurezza interna), da entità esterne, imposte da standard oppure da regolamentazioni nazionali o internazionali.
La sicurezza di un sistema o di un insieme di asset dipende solo parzialmente dalla forza dei singoli meccanismi di protezione isolati
l'uno dall'altro; infatti, dipende anche dall'abilità di questi meccanismi di lavorare continuamente in sinergia per provvedere una 
protezione olistica.\hfill\break
Moon Cloud è un framework di \textit{Security Assurance} il quale garantisce che un sistema ICT soddisfi certi requisiti 
prestabiliti da appropriate politiche e procedure precedentemente definite. Una \textit{Security Compliance Evaluation} è un processo 
di verifica a cui un target è sottoposto e il cui risultato deve soddisfare i requisiti richiesti da standard e politiche. A partire da questi 
processi di controllo, che devono a loro volta essere affidabili, si ottengono delle evidence; queste ultime possono essere raccolte 
monitorando l'attività del target oppure, come già menzionato, sottoponendo il target a scenari critici o di testing.\hfill\break
In particolare, una Security Compliance Evaluation è un processo di verifica dell'uniformità di un certo target a una o più politiche 
attraverso una serie di Controlli che a seconda delle caratteristiche e proprietà del target, può avere successo o meno. Di 
conseguenza se un target supera tutti i Controlli a cui è sottoposto allora significa che rispetta la politica scelta.
\begin{figure}[ht!]
    \includegraphics[scale=0.53]{images/Security_Compliance_Evaluation.jpg}
    \caption[Security Compliance Evaluation]{Security Compliance Evaluation}
    \label{fig:Security_Compliance_Evaluation}
\end{figure}
\hfill\break
\break
Moon Cloud implementa il processo di Security Compliance Evaluation in Figura \ref{fig:Security_Compliance_Evaluation} usando Controlli 
di monitoraggio o di test personalizzabili. Inoltre, è dotato delle seguenti caratteristiche, le quali vanno a completare i requisiti 
elencati nella Sezione \ref{list:security-assurance-fondamentals}.
\begin{itemize}
    \item Moon Cloud implementa un sistema di Security Assurance Evidence-based continuo, implementato come processo di Compliance,
    basato su politiche custom o standard; inoltre, presenta una visione olistica dello stato di sicurezza di un dato sistema.
    \item Moon Cloud permette di schedulare e configurare delle ispezioni automatiche, grazie all'inventario di asset protetto e senza
    l'intervento dell'uomo.
    \item Moon Cloud Evaluation Engine può ispezionare dall'interno un sistema, gestendo così delle minaccie interne; permettendo anche 
    reazioni rapide a incidenti di sicurezza e veloci rimedi, grazie alla raccolta continua di evidence.
\end{itemize}
In generale, Moon Cloud gestisce i processi di Evaluation attraverso un set di \textit{Execution Cluster}; ognuno dei quali gestisce 
ed esegue un set di \textit{probe} che collezionano le evidence necessarie per effettuare i processi di valutazione. 
Tutte le attività di collezione sono eseguite dalla probe, ognuno dei quali è uno script Python fornito come una singola immagine di Docker, 
che viene inizializzata quando è triggerata una Evaluation ed è distrutta quando il processo di Evaluation è terminato.\hfill\break
Accedendo alla piattaforma di Moon Cloud, l'utente può definire le proprie politiche di sicurezza e attività di Evaluation come 
una serie di Controlli di sicurezza e altre politiche predefinite. Una volta che una politica viene definita, l'utente può 
decidere quando schedulare l'Evaluation; e nel momento in cui un processo di Evaluation viene inizializzato, tutti i Controlli e/o le 
politiche legate ad essa, vengono eseguiti e i risultati, raccolti dalla probe, vengono memorizzati e restituiti all'utente. 
A questo punto l'utente può accedere a questi risultati a diversi gradi di precisione: una visione sommaria e generale di tutte le 
politche implentate e dello stato generale del sistema di sicurezza, al risultato di una specifica politica oppure alle evidence 
raccolte per una Evaluation.
\vspace{0.5 cm}
\hfill\break
Per poter rendere ancora più intuitivo e semplice da utilizzare un sistema di questa importanza, si è pensato d'introdurre un sistema 
che possa raccomandare agli utenti, in base agli asset che vogliono proteggere e monitorare, una serie di Evaluation o politiche da 
applicare in quei casi; questo permette anche a utenti meno esperti di poter configurare in modo rapido ed efficiente meccanismi di 
protezione da minacce. Un sistema di raccomandazione permette di selezionare all’interno di un ampio catalogo, un numero limitato di prodotti 
personalizzati sulla base delle preferenze dell’utente attivo. La ricerca in questo ambito si è sempre concentrata sulla qualità delle 
raccomandazioni di questi sistemi, tralasciando un aspetto fondamentale: la fiducia che un utente deve avere verso questi ultimi. 
E ciò è ottenibile se si è il più possibile trasparenti nei processi che portano alla nascita dei suggerimenti, partendo da questo 
si può ottenere l'ambita fiducia da parte degli utenti.
