\chapter{Introduzione}
In questo capitolo verrà descritto in modo più approfondito 
il funzionamento della piattaforma MoonCloud e unitamente al 
motivo dell'implementazione della soluzione proposta.

\section{Perchè MoonCloud?}
La diffusione di sistemi ICT (\textit{Information and Communications Technology})
nella maggiorparte degli ambienti lavorativi e privati in termini di servizi offerti,
automazione di processi e incremento delle performance. L'uso di questa tecnologia 
ha assunto importanza a partire dagli anni novanta come effetto del boom di Internet.
Oggi le professionalità legate all'ICT crescono in numero e si evolvono per 
specificità, per operare in ambienti fortemente eterogenei ma sempre più interconnessi 
fra di loro come cloud computing, social newtwork, marketing digitale, IoT, realtà virtuale,
ecc.

Il prezzo che paghiamo per i benefici di queste tecnologie è dato dall'incremento di 
violazione di sicurezza, che oggigiorno preoccupa tutte le aziende, e di conseguenza anche i 
loro clienti, con l'incremento del rischio di fallimento per i servizi più importanit, 
violazioni della privacy e furto di dati.
\newline
Il mercato sta lentamente notando che non è l'inadequamento tecnologico dei sistemi di sicurezza
che incrementa il rischio di furti di dati o violazioni di sicurezza; piuttosto, la mal configurazione
e errata integrazione di questi sistemi nei processi di business sono la base per 
i furti e le violazioni. \cite{cloud-Platform-for-ICT-Security-Governance}

Per questo motivo anche se vengono usati i sistemi di sicurezza e di controllo migliori non è possibile 
garantire la sicurezza; ma è necessario implementare un processo continuo di diagnostica che verifica che 
controlli sono configurati in modo corretto e il loro comportamento è quello aspettato.

Security assesment diventa allora un aspetto importante specialmente negli ambienti cloud e IoT. Questo 
assesment deve essere fatto in modo continuo e olistico, per correlare le prove raccolte da sempre maggiori 
meccanismi di protezione
\newline
MoonCloud è una soluzione PaaS (Platform as a Service) che fornice una piattaforma B2B innovativa per verifiche, 
diagnostiche e monitoraggio dell'adeguatezza dei sistemi ICT rispetto alle politiche di sicurezza, in modo continuo 
e su larga scala.
MoonCloud supporta una semplice ed efficente \textit{ICT security governance}, dove le politiche di sicurezza possono
essere definite dalle compagnie stesse (a partire da un semplice controllo sulle vulnerabilità a linee guida di
sicurezza interna), da entità esterne, imposte da standard oppure da regolamentazioni nazionali/internazionali.
\newline
La sicurezza di un sistema o di un insieme di asset dipende solo parzialmente dalla forza dei
singoli meccanismi di protezione isolati l'uno dall'altro; infatti, dipende anche dall'abilità di questi meccanismi 
di lavorare continuamente in sinergia per provvedere a una protezione olistica.
In più, quando i sistemi cloud e i servizi IoT sono coinvolti, le dinamiche di questi servizi e la loro rapida 
evoluzione rende il controllo dei processi all'interno dell'azienda e le politiche di sicurezza 
più complesse e prone ad errori.
\newline
I requisiti ad alto livello fondamentali per poter garantire le security assurance sono:
\begin{itemize}
	\item \textbf{sistema olistico}
	\item \textbf{monitoraggio continuo ed efficiente}
	\item \textbf{inventario degli asset}
	\item \textbf{rezioni rapide a incidenti di sicurezza}
	\item \textbf{saper porre rimedio}
\end{itemize}