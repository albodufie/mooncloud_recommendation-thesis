\chapter{Introduzione}
In questo capitolo verrà descritto in modo più approfondito 
il funzionamento della piattaforma Moon Cloud e unitamente al 
motivo dell'implementazione della soluzione proposta.

\section{Perchè Moon Cloud?}
La diffusione di sistemi ICT (\textit{Information and Communications Technology})
nella maggiorparte degli ambienti lavorativi e privati in termini di servizi offerti,
automazione di processi e incremento delle performance. L'uso di questa tecnologia 
ha assunto importanza a partire dagli anni novanta come effetto del boom di Internet.
Oggi le professionalità legate all'ICT crescono in numero e si evolvono per 
specificità, per operare in ambienti fortemente eterogenei ma sempre più interconnessi 
fra di loro come cloud computing, social newtwork, marketing digitale, IoT, realtà virtuale,
ecc.

Il prezzo che paghiamo per i benefici di queste tecnologie è dato dall'incremento di 
violazione di sicurezza, che oggigiorno preoccupa tutte le aziende, e di conseguenza anche i 
loro clienti, con l'incremento del rischio di fallimento per i servizi più importanit, 
violazioni della privacy e furto di dati.
\newline
Il mercato sta lentamente notando che non è l'inadequamento tecnologico dei sistemi di sicurezza
che incrementa il rischio di furti di dati o violazioni di sicurezza; piuttosto, la mal configurazione
e errata integrazione di questi sistemi nei processi di business sono la base per 
i furti e le violazioni. \cite{cloud-Platform-for-ICT-Security-Governance}

Per questo motivo anche se vengono usati i sistemi di sicurezza e di controllo migliori non è possibile 
garantire la sicurezza; ma è necessario implementare un processo continuo di diagnostica che verifica che 
controlli sono configurati in modo corretto e il loro comportamento è quello aspettato.

Security assesment diventa allora un aspetto importante specialmente negli ambienti cloud e IoT. Questo 
assesment deve essere fatto in modo continuo e olistico, per correlare le prove raccolte da sempre maggiori 
meccanismi di protezione
\newline
Moon Cloud è una soluzione PaaS (Platform as a Service) che fornice una piattaforma B2B innovativa per verifiche, 
diagnostiche e monitoraggio dell'adeguatezza dei sistemi ICT rispetto alle politiche di sicurezza, in modo continuo 
e su larga scala.
Moon Cloud supporta una semplice ed efficente \textit{ICT security governance}, dove le politiche di sicurezza possono
essere definite dalle compagnie stesse (a partire da un semplice controllo sulle vulnerabilità a linee guida di
sicurezza interna), da entità esterne, imposte da standard oppure da regolamentazioni nazionali/internazionali.
\newline
La sicurezza di un sistema o di un insieme di asset dipende solo parzialmente dalla forza dei
singoli meccanismi di protezione isolati l'uno dall'altro; infatti, dipende anche dall'abilità di questi meccanismi 
di lavorare continuamente in sinergia per provvedere a una protezione olistica.
In più, quando i sistemi cloud e i servizi IoT sono coinvolti, le dinamiche di questi servizi e la loro rapida 
evoluzione rende il controllo dei processi all'interno dell'azienda e le politiche di sicurezza 
più complesse e prone ad errori.
\newline
I requisiti ad alto livello fondamentali per poter garantire le security assurance sono:
\begin{description}
	\item[sistema olistico] è richiesta una visione globale e pulita dello stautus dei sistemi di sicurezza; 
	inoltre è cruciale distribuire lo sforzo degli specialisti in sicurezza per migliorare il processo e le 
	politiche messe in atto. Si parte da della valutazioni fatte manualmente a quella semi-automatiche che
	ispezionano i meccanismi di sicurezza. 
	\item[monitoraggio continuo ed efficiente] è necessario un controllo continuo che valuti l'efficenza dei 
	sistemi di sicurezza per ridurre l'impatto dell'errore umano, soprattutto dal punto di vista organizzativo.
	La mancata configurazione dovuta al cambiamento dell'ambiente, la coesistenza di componenenti in conflitto: 
	sono scenari che richiedono un monitoraggio e un aggiornamento continuo.
	\item[singolo punto managment] avere un solo punto in cui gestire tutti gli aspetti relativi alla sicurezza,
	permette di avere sotto controllo le politiche di sicurezza, Inoltre disporre di un inventario degli asset 
	da proteggere, così da poter conoscere quali protezioni applicare.
	\item[reazioni rapide a incidenti di sicurezza] spesso la reazione ad incidenti di sicurezza è ritardata 
	da due fattori: il tempo richiesto per rilevare l'incidente e il tempo per analizzare il motivo dell'accaduto.
\end{description}


Moon Cloud è basato su una tecnica di security assurance garantendo che tutti le attività aziendali si compiano
seguendo i requisiti prestabiliti da appropriate politiche e procedure.

Una security compliance evaluation è il processo di verifica che un target è sottoposto il cui risultato deve 
soddisfare i richiesti standard e politiche. Da questi processi di verifica, che devono a loro volta essere 
affidabili, si ottendo delle evidence (prove); queste ultime possono essere raccolte monitorando l'attività
del target oppure, come già menzionato, sottoponendo il target a scenari
critici o testing.
In particolare, una security compliance evaluation è un processo che verifica la compliance (uniformità)
di un certo target a una o più politiche; vengono eseguiti tutti i controlli e produce un valore booleano per
le politiche, in base al valore booleano associato ad ogni controllo.