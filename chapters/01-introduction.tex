\chapter{Introduzione}
In questo capitolo verrà descritto in modo più approfondito 
il funzionamento della piattaforma MoonCloud e unitamente al 
motivo dell'implementazione della soluzione proposta.

\section{Perchè MoonCloud?}
La diffusione di sistemi ICT (\textit{Information and Communications Technology})
nella maggiorparte degli ambienti lavorativi e privati in termini di servizi offerti,
automazione di processi e incremento delle performance. L'uso di questa tecnologia 
ha assunto importanza a partire dagli anni novanta come effetto del boom di Internet.
Oggi le professionalità legate all'ICT crescono in numero e si evolvono per 
specificità, per operare in ambienti fortemente eterogenei ma sempre più interconnessi 
fra di loro come cloud computing, social newtwork, marketing digitale, IoT, realtà virtuale,
ecc.

Il prezzo che paghiamo per i benefici di queste tecnologie è dato dall'incremento di 
violazione di sicurezza, che oggigiorno preoccupa tutte le aziende, e di conseguenza anche i 
loro clienti, con l'incremento del rischio di fallimento per i servizi più importanit, 
violazioni della privacy e furto di dati.

Il mercato sta lentamente notando che non è l'inadequamento tecnologico dei sistemi di sicurezza
che incrementa il rischio di furti di dati o violazioni di sicurezza; piuttosto, la mal configurazione
e errata integrazione di questi sistemi nei processi di business sono la base per 
i furti e le violazioni. \cite{cloud-Platform-for-ICT-Security-Governance}

Per questo motivo anche se vengono usati i sistemi di sicurezza e di controllo migliori non è possibile 
garantire la sicurezza; ma è necessario implementare un processo continuo di diagnostica che verifica che 
controlli sono configurati in modo corretto e il loro comportamento è quello aspettato.

Security assesment diventa allora un aspetto importante specialmente negli ambienti cloud e IoT. Questo 
assesment deve essere fatto in modo continuo e olistico