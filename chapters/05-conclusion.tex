\chapter{Conclusioni}\label{chp:05-conclusion}
La soluzione proposta in questa tesi vuole introdurre un sistema di raccomandazione in un mondo in cui spesso non vengono 
introdotti perché popolato da utenti esperti che non ne avrebbero bisogno; in questo modo si da la possibilità a un 
maggior numero di utenti di accedere a servizi su un sistema Cloud di Security Assurance, come 
Moon Cloud, in totale sicurezza e affidabilità.\hfill\break
Con questo lavoro è stato possibile studiare e approfondire il linguaggio di programmazione Python, unitamente al framework 
Django per la realizzazione di applicativi web e la tecnologia Docker per il rilascio in ambienti isolati e indipendenti 
di software; inoltre sono stati approfonditi i temi legati ai Recommendation System e al mondo del machine learning.
%
\section{Sviluppi futuri}
Il sistema di raccomandazione ideato in questo progetto è molto basilare ma offre le più disparate e numerose opportunità 
di crescita ad esempio una possibile modifica sarebbe quella d'introdurre un sistema di valutazione delle Evaluation o 
dei Controlli da parte dell'utente, e incrementare la precisione del sistema di raccomandazione tenendo conto anche di
queste valutazioni.
% Presentazione dei risultati e conclusioni: la presentazione dei risultati dovrebbe consistere in una 
% descrizione tecnica dei risultati raggiunti, unitamente a un commento critico e a un’analisi della rispondenza agli 
% obiettivi iniziali (si consiglia pertanto di motivare 
% la rilevanza dei risultati e l’eventuale scostamento dagli obiettivi iniziali). La sezione relativa ai risultati dovrebbe 
% infine contenere una sintesi critica e un giudizio sull’esperienza effettuata, che renda conto di aspetti positivi e negativi
% per il tirocinante e per l’ente ospitante, del valore formativo, professionale e umano, così via.
