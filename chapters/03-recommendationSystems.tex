\chapter{Collaborative filtering}
\label{chp:03-recommendationSystems}
In questo capitolo verranno approfonditi gli algoritmi di raccomandazione implementati nella soluzione, mostrando le porzioni di 
codice e spiegando i vari passagi che portano ad ottenere delle raccomandazioni.
% - nel capitolo dei recommendation system parlare dei collaborative filter in modo più dettagliato e legato al progetto -


\section{Memory-based} 
I Filtri Collaborativi Memory-based sono stati introdotti per via delle osservazioni che vennero fatte dalla comunità, dicenddo che
gli utenti si fidano maggiormente delle raccomandazioni di altri che la pensano allo stesso modo. Questi metodi mirano a calcolare 
le relazioni tra utenti e item attraverso lo schema dei vicini che identifica sia coppie di item che tendono ad essere usati insieme 
o hanno un grado di similarità alto o utenti con uno storico di item usati simile. \cite{taxonomy-of-recommender-agents-on-the-internet}
Questi approcci divennero molto famosi grazie alla loro semplicità di implementazione, molto intuitivi, non necessitano di operazioni
di training sui dati e regolazione di molti parametri, inoltre l'utente può capire la ragione che sta dietro ogni raccomandazione. 

Questa categoria di sistemi di raccomandazione sono definiti anche \textit{Neighborhood-based} e possono essere ulteriormente classificati 
in due sottocategorie:


\subsection{User-based filtering} 
Questo sistema, definiti anche con l'acronimo UB-CF (\textit{User-based Collaborative Filter}) basa tutto il suo funzionamento sulla 
comunità di utenti, maggiore è la sua dimensione e l'attività degli utenti con item o servizi e migliori potranno essere le 
raccomandazioni. Questo algoritmo fornisce dei suggerimenti ad un utente sulla base di uno o più vicini, e la similarità tra utenti può
essere determinata sulla base degli item che l'utente ha utilizzato o valutato.

Molti di questi approcci possono essere generalizzati dall'algoritmo definito dai seguenti step:
\begin{enumerate}
	\item Specificare qual'è l'utente a cui si vuole applicare l'algoritmo di raccomandazione e recuperare quali utenti possono 
	avere dato valutazioni o usato item simili al utente target. Piuttosto che recuperare tutti gli utenti, per velocizzare l'esecuzione
	dell'algoritmo, è possibile selezionare soltanto un gruppo di utenti in modo casuale oppure associare dei valori di similarità tra 
	tutti gli utenti e confrontando questi valori con quello dell'utente target, selezionare i relativi utenti che superano una soglia
	scelta, oppure utilizzare tecniche di clustering.
	\item Estrarre quegli item a cui l'utente target non ha mai interagito e per questo motivo gli possono interessare, e mostrarli 
	sottoforma di raccomandazioni.
\end{enumerate}

\begin{figure}[ht!]
	\centering
	\includegraphics[scale=0.5]{images/UB_CF_ex.png}
	\caption{Esempio di applicazione di un sistema di raccomandazione User-based}
	\label{fig:UB_CF}
\end{figure}

Questi approcci sono facilmente implementabili, indipendenti dal contesto in cui sono applicati e possono essere più accurati rispetto
a tecniche basate sul Content-based filtering; dall'altra parte all'aumentare del numero di utenti che vado a considerare per fare le 
raccomandazioni migliore è la precisione di questo processo ma anche è maggiore il costo in termini di tempo.  


\subsection{Item-based filtering} 
Quando viene applicato per milioni di utenti e item, l'algoritmo UB-CF non è molto efficente, per via della complessa computazione della 
ricerca di utenti simili; così in alternativa è stato introdotto l'algoritmo di filtraggio Item-based, definito anche IB-CF 
(\textit{Item-based Collaborative Filter}) dove piuttosto che effetuare il confronto tra utenti simili, viene fatto un confronto tra 
gli item dell'utente a cui si vuole raccomandare e i possibili item simili.

Questi sistemi sono estremamente simili ai sistemi di raccomandazione Content-based, e identificano item simili in base a come utenti gli
hanno usati nel passato. \cite{item-based-collaborative-filtering}

\begin{figure}[ht!]
	\centering
	\includegraphics[scale=0.5]{images/IB_CF_ex.png}
	\caption{Esempio di applicazione di un sistema di raccomandazione Item-based}
	\label{fig:IB_CF}
\end{figure}


\subsection{Hybrid filtering} 
